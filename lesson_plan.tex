\chapter{แผนการสอน}
\section{คำอธิบายรายวิชาและวัตถุประสงค์ที่ระบุไว้ในหลักสูตร}

แนวความคิดเรื่องการเขียนโปรแกรม ขั้นตอนวิธีในการแก้ไขปัญหา การสร้างคำสั่งสำหรับเขียนขั้นตอนวิธีการ เขียนผังงาน นิพจน์ คำสั่งในการเขียนโปรแกรม หลักไวยากรณ์ของภาษาโปรแกรมระดับสูง การเขียนโปรแกรมสมัยใหม่ การทดสอบ การแก้ไขโปรแกรม การติดตั้ง และการเขียนเอกสารประกอบโปรแกรม

Concept of programming, algorithm to solve the problem, flowchart, expression and instruction, high-level language syntax, modern programming, testing, debugging, installation and software documentation

\section{วัตถุประสงค์ของวิชา}

มีจุดมุ่งหมายให้นักศึกษาได้มีพื้นฐานความรู้ความเข้าใจในหลักการเขียนโปรแกรมคอมพิวเตอร์เบื้องต้นด้วยภาษา Python ส่วนประกอบต่างๆ ของโปรแกรมคอมพิวเตอร์และภาษา Python สามารถเขียนโปรแกรมคอมพิวเตอร์อย่างง่ายด้วยภาษา Python ได้ตามการวิเคราะห์และออกแบบขั้นตอนการทำงานของโปรแกรมอย่างมีระบบ และมีความรู้ความเข้าใจในการเขียนโปรแกรมแบบมีเงื่อนไขเพื่อการตัดสินใจ การเขียนคำสั่งเพื่อการทำงานซ้ำ และโมดูลส่วนเสริมต่างๆ ของโปรแกรมภาษา Python เพื่อเรียนรู้เรื่องการเขียนโปรแกรมแบบฟังก์ชันและการเขียนโปรแกรมเชิงวัตถุได้

\section{เนื้อหาวิชา}

\subsection{สัปดาห์ที่ 1}

\begin{description}

\item[ผู้สอน] จันทวรรณ ปิยะวัฒน์
\item[จำนวนชั่วโมงบรรยาย] 2
\item[จำนวนชั่วโมงปฎิบัติ] 2

\item[หัวข้อ/รายละเอียด] \hfill \\
\underline{เค้าโครงวิชา}

\begin{itemize}
\item วัตถุประสงค์รายวิชา
\item รายละเอียดเนื้อหาวิชา
\item การวัดผลและการประเมินผล
\item เงื่อนไขและข้อตกลงอื่น
\item วิธีการเรียนการสอน
\item เว็บไซต์และหนังสืออ่านประกอบ
\end{itemize}

\underline{ระบบจัดการการเรียนรู้ (ClassStart.org)}
\begin{itemize}
\item ระบบในภาพรวม
\item การสมัครสมาชิก
\item การเข้าห้องเรียนออนไลน์ของรายวิชา
\item การใช้งานระบบ
\item การเข้าอ่านเอกสารการสอนและคลิป
\item การส่งแบบฝึกหัดทางออนไลน์
\item การทำข้อสอบออนไลน์
\item การตรวจสอบคะแนนเก็บ
\item การบันทึกการเรียนรู้ (Reflections)
\item การสื่อสารออนไลน์
\end{itemize}

\underline{เว็บไซต์ Code.org}
\begin{itemize}
\item การสมัครสมาชิก
\item ฝึกการเขียนโปรแกรมง่าย ๆ (Game-based Learning) แบบ Block-based Programming
\end{itemize}

\item[กิจกรรมการเรียนการสอน/สื่อที่ใช้] \hfill
\begin{itemize}[leftmargin=0pt]
\item บรรยาย
\item ปฎิบัติการใช้ระบบ ClassStart.org
\item ปฎิบัติการเขียนโปรแกรมทางออนไลน์ที่ Code.org
\end{itemize}

\end{description}

\subsection{สัปดาห์ที่ 2}
