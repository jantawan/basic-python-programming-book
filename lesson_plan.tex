\chapter{แผนการสอน 1/2563}

\section*{ข้อมูลรายวิชา}
\definecolor{myblue}{cmyk}{0.80, 0.13, 0.14, 0.04, 1.00}
\begin{tcolorbox}[breakable,enhanced,fonttitle=\bfseries,colback=myblue!05,colframe=myblue]
\begin{itemize}
\item 477-201 การเขียนโปรแกรมคอมพิวเตอร์ (Section 01)
\item อังคาร และ พฤหัสบดี เวลา 8:00-9:50 น.
\item ห้องเรียนออนไลน์ \url{https://classstart.org} หมายเลขประจำวิชา: \textbf{51390}
\item ห้องประชุมออนไลน์ Microsoft Teams \url{https://bit.ly/31F32RF}
\item ไลน์กลุ่ม: ติดต่อไลน์ไอดี jantawan.n
\item ห้องคอมพิวเตอร์ FMS 3503 (สำหรับนักศึกษาที่ไม่มีอุปกรณ์คอมพิวเตอร์ใช้)
\end{itemize}
\end{tcolorbox}

\section*{ข้อมูลผู้สอน}
\definecolor{myblue}{cmyk}{0.80, 0.13, 0.14, 0.04, 1.00}
\begin{tcolorbox}[breakable,enhanced,fonttitle=\bfseries,colback=myblue!05,colframe=myblue]
\begin{itemize}
\item ดร.จันทวรรณ ปิยะวัฒน์
\item Line ID: jantawan.n (ติดต่อได้ตลอดในวันและเวลาราชการ)
\item Email: jantawan.n@psu.ac.th
\end{itemize}
\end{tcolorbox}

\section*{คำอธิบายรายวิชาและวัตถุประสงค์ที่ระบุไว้ในหลักสูตร}

\definecolor{myblue}{cmyk}{0.80, 0.13, 0.14, 0.04, 1.00}

\begin{tcolorbox}[breakable,enhanced,fonttitle=\bfseries,colback=myblue!05,colframe=myblue]
แนวความคิดเรื่องการเขียนโปรแกรม ขั้นตอนวิธีในการแก้ไขปัญหา การสร้างคำสั่งสำหรับเขียนขั้นตอนวิธีการ เขียนผังงาน นิพจน์ คำสั่งในการเขียนโปรแกรม หลักไวยากรณ์ของภาษาโปรแกรมระดับสูง การเขียนโปรแกรมสมัยใหม่ การทดสอบ การแก้ไขโปรแกรม การติดตั้ง และการเขียนเอกสารประกอบโปรแกรม

Concept of programming, Algorithm to solve the problem, Flowchart, Expression and instruction, High-level language syntax, Modern programming, Testing, Debugging, Installation and software documentation
\end{tcolorbox}

\section*{วัตถุประสงค์ของวิชา}

\begin{tcolorbox}[breakable,enhanced,fonttitle=\bfseries,colback=myblue!05,colframe=myblue]
มีจุดมุ่งหมายให้นักศึกษาได้มีพื้นฐานความรู้ความเข้าใจในหลักการเขียนโปรแกรมคอมพิวเตอร์เบื้องต้นด้วยภาษา Python ส่วนประกอบต่างๆ ของโปรแกรมคอมพิวเตอร์และภาษา Python สามารถเขียนโปรแกรมคอมพิวเตอร์อย่างง่ายด้วยภาษา Python ได้ตามการวิเคราะห์และออกแบบขั้นตอนการทำงานของโปรแกรมอย่างมีระบบ และมีความรู้ความเข้าใจในการเขียนโปรแกรมแบบมีเงื่อนไขเพื่อการตัดสินใจ การเขียนคำสั่งเพื่อการทำงานซ้ำ และโมดูลส่วนเสริมต่างๆ ของโปรแกรมภาษา Python เพื่อเรียนรู้เรื่องการเขียนโปรแกรมแบบฟังก์ชันและการเขียนโปรแกรมเชิงวัตถุได้
\end{tcolorbox}
\vspace{1.5cm}

\section*{เนื้อหาวิชา}

% \lesson command definition

\definecolor{mybrown}{RGB}{128,64,0}

\newcommand{\lesson}[3]{
% \subsection*{สัปดาห์ที่ {#1}}
\begin{tcolorbox}[breakable,enhanced,fonttitle=\bfseries,title=สัปดาห์ที่ {#1},colback=myblue!05,colframe=myblue]
\begin{description}
\vspace{0.5cm}
\item[ผู้สอน] จันทวรรณ ปิยะวัฒน์
\item[จำนวนชั่วโมงบรรยาย] 2
\item[จำนวนชั่วโมงปฎิบัติ] 2
\item[หัวข้อ/รายละเอียด] \hfill \\
{#2}
\item[กิจกรรมการเรียนการสอน/สื่อที่ใช้] \hfill
\begin{itemize}[leftmargin=0pt]
{#3}
\end{itemize}
\vspace{1cm}
\end{description}
\end{tcolorbox}
% \vspace{1cm}
%\newpage
}

% --

\lesson{1}
{
\underline{เค้าโครงวิชา}

\begin{itemize}
\item วัตถุประสงค์รายวิชา
\item รายละเอียดเนื้อหาวิชา
\item การวัดผลและการประเมินผล
\item เงื่อนไขและข้อตกลงอื่น
\item วิธีการเรียนการสอน
\item เว็บไซต์และหนังสืออ่านประกอบ
\end{itemize}

\underline{แพลตฟอร์มการการเรียนรู้ออนไลน์ \textbf{ClassStart}}
\begin{itemize}
\item ระบบในภาพรวม
\item การสมัครสมาชิก
\item การเข้าห้องเรียนออนไลน์ของรายวิชา
\item การใช้งานระบบ
\item การเข้าอ่านเอกสารการสอนและคลิป
\item การส่งแบบฝึกหัดทางออนไลน์
\item การทำข้อสอบออนไลน์
\item การตรวจสอบคะแนนเก็บ
\item การบันทึกการเรียนรู้ (Reflections)
\item การสื่อสารออนไลน์
\end{itemize}

\underline{ระบบ VDO conference เพื่อการสอนสด \textbf{Microsoft Teams}}
\begin{itemize}
\item เข้าร่วม VDO conference room
\item ฝึกใช้ระบบ VDO conference
\end{itemize}

\underline{บทที่ 1 ความรู้เบื้องต้นเกี่ยวกับ Python}
\begin{itemize}
\item Python คืออะไร
\item  Python ทำงานอย่างไร
\item  อัลกอริทึมและผังงาน
\item  การติดตั้งโปรแกรม Python Runtime
\end{itemize}
}
{
\item ปฎิบัติการใช้ระบบ \textbf{ClassStart} (\url{https://ClassStart.org})
\item ปฎิบัติการใช้ระบบ \textbf{Microsoft Teams} (\url{https://bit.ly/31F32RF})
\item บรรยายและยกตัวอย่างการเขียนโปรแกรม
\item  ถามตอบในชั้นเรียน
\item  ฝึกการเขียนผังงาน
\item  ปฏิบัติการติดตั้ง Python Runtime
\item  ปฏิบัติการเขียนโปรแกรม
\item  บันทึกการเรียนรู้
}


\lesson{2}
{
\underline{บทที่ 2 ส่วนประกอบของ Python}
\begin{itemize}
\item ตัวแปร
\item ประเภทของข้อมูล
\item การคำนวณ
\item Expressions และ Statements
\item Comments
\item Source Code
\item คำสั่ง  \texttt{print()}
\item คำสั่ง  \texttt{input()}
\end{itemize}
\underline{บทที่ 3 ประโยคเงื่อนไข}
\begin{itemize}
\item Boolean Expressions
\end{itemize}

}
{
\item ทดสอบทบทวนความรู้
\item บรรยายและยกตัวอย่างการเขียนโปรแกรม
\item ถามตอบในชั้นเรียน
\item ปฏิบัติการเขียนโปรแกรม
\item บันทึกการเรียนรู้
}

\lesson{3}
{
\underline{บทที่ 3 ประโยคเงื่อนไข (ต่อ)}
\begin{itemize}
\item if, elif, else
\item Chained Expressions
\item Nested Expressions
\end{itemize}
}
{
\item ทดสอบทบทวนความรู้
\item  บรรยายและยกตัวอย่างการเขียนโปรแกรม
\item  ถามตอบในชั้นเรียน
\item  ปฏิบัติการเขียนโปรแกรม
\item  บันทึกการเรียนรู้
}
\lesson{4}
{
\underline{บทที่ 4 การเขียนและใช้งานฟังก์ชัน}
\begin{itemize}
\item การเรียกใช้ฟังก์ชัน
\item การเรียกใช้โมดูล
\item ฟังก์ชันซ้อน
\item การสร้างฟังก์ชัน
\item การคืนค่าของฟังก์ชัน
\item การเขียนโปรแกรมเชิงฟังก์ชัน
\item การเขียนคำอธิบายโปรแกรม
\end{itemize}
}
{
\item  บรรยายและยกตัวอย่างการเขียนโปรแกรม
\item  ถามตอบในชั้นเรียน
\item  ปฏิบัติการเขียนโปรแกรม
\item  บันทึกการเรียนรู้
}

\lesson{5}
{
\underline{บทที่ 5 การใช้ประโยคสั่งทำงานวนซ้ำ}
\begin{itemize}
\item ฟังก์ชัน \texttt{range()}
\item คำสั่ง \texttt{for}
\item คำสั่ง \texttt{while}
\item คำสั่ง \texttt{break}
\item ฟังก์ชันที่เรียกตัวเอง
\end{itemize}

}
{
\item  บรรยายและยกตัวอย่างการเขียนโปรแกรม
\item  ถามตอบในชั้นเรียน
\item  ปฏิบัติการเขียนโปรแกรม
\item  บันทึกการเรียนรู้
}
\lesson{6}
{
\underline{Python project 1}
\begin{itemize}
\item ดำเนินการคิดและพัฒนา Project
\item นำเสนอ Project
\end{itemize}
}
{
\item  ปฏิบัติการเขียนโปรแกรม
\item  นำเสนอผลงาน
}
\lesson{7}
{
\underline{บทที่ 6 การใช้งาน String}
\begin{itemize}
\item ฟังก์ชัน \texttt{len()}
\item การเดินทางตามตัวชี้ของ String
\item การตัดคำใน String
\item โครงสร้างข้อมูลที่เปลี่ยนแปลงไม่ได้
\item การค้นหาตัวอักษรใน String
\item String Methods
\item การใช้  \texttt{in}
\item การเปรียบเทียบ String
\item การจัดวางรูปแบบของ String

\end{itemize}

}
{
\item ทดสอบทบทวนความรู้
\item  บรรยายและยกตัวอย่างการเขียนโปรแกรม
\item  ถามตอบในชั้นเรียน
\item  ปฏิบัติการเขียนโปรแกรม
\item  บันทึกการเรียนรู้
}
\lesson{8}
{
\underline{บทที่ 7 ลิสต์ (List)}
\begin{itemize}
\item การเข้าถึงค่าในลิสต์
\item การแบ่งข้อมูลในลิสต์
\item การใช้ in กับลิสต์
\item การเดินทางในลิสต์
\item ตัวเนินการของลิสต์
\item List Methods
\item Map, reduce, and filter
\item Lists กับ String
\item Objects กับ values
\end{itemize}
}
{
\item ทดสอบทบทวนความรู้
\item  บรรยายและยกตัวอย่างการเขียนโปรแกรม
\item  ถามตอบในชั้นเรียน
\item  ปฏิบัติการเขียนโปรแกรม
\item  บันทึกการเรียนรู้
}
\lesson{9}
{
\underline{บทที่ 8 ดิกชันนารี (Dictionary)}
\begin{itemize}
\item การอ่านค่าใน Dictionary
\item การหาค่าของ Key ใน Dictionary
\item Dictionary and List
\item ฟังก์ชันที่รับ Parameters ได้ไม่จำกัด
\end{itemize}
}
{
\item ทดสอบทบทวนความรู้
\item  บรรยายและยกตัวอย่างการเขียนโปรแกรม
\item  ถามตอบในชั้นเรียน
\item  ปฏิบัติการเขียนโปรแกรม
\item  บันทึกการเรียนรู้
}
\lesson{10}
{
\underline{บทที่ 9 ทูเบิล (Tuple)}
\begin{itemize}
\item ความหมายของ Tuple
\item การสลับค่าของ Tuple
\item การเก็บค่าการดำเนินการใน Tuple
\item Dictionary และ Tuple
\end{itemize}
}
{
\item ทดสอบทบทวนความรู้
\item  บรรยายและยกตัวอย่างการเขียนโปรแกรม
\item  ถามตอบในชั้นเรียน
\item  ปฏิบัติการเขียนโปรแกรม
\item  บันทึกการเรียนรู้
}
\lesson{11}
{
\underline{บทที่ 10 การจัดการไฟล์ (Files)}
\begin{itemize}
\item การทำงานกับ Directories
\item การเปิดไฟล์
\item การอ่านไฟล์
\item การจัดการข้อผิดพลาด
\item ฐานข้อมูลแบบ Key-Value
\item การเรียกใช้โปรแกรมอื่น
\end{itemize}
}
{
\item ทดสอบทบทวนความรู้
\item  บรรยายและยกตัวอย่างการเขียนโปรแกรม
\item  ถามตอบในชั้นเรียน
\item  ปฏิบัติการเขียนโปรแกรม
\item  บันทึกการเรียนรู้
}
\lesson{12}
{
\underline{บทที่ 11 Object-Oriented Programming}
\begin{itemize}
\item คลาสและออบเจ็กต์
\item การสร้างคลาส
\item การสร้างออบเจ็กต์
\item ฟังก์ชัน \texttt{\_\_init\_\_()}
\item การสร้างเมธอดของออบเจ็กต์
\item การแก้ไขค่าแอตทริบิวต์ของออบเจ็กต์
\item การลบแอตทริบิวต์ของออบเจ็กต์
\item การลบออบเจ็กต์
\item การสืบทอดคลาส
\end{itemize}
}
{
\item ทดสอบทบทวนความรู้
\item  บรรยายและยกตัวอย่างการเขียนโปรแกรม
\item  ถามตอบในชั้นเรียน
\item  ปฏิบัติการเขียนโปรแกรม
\item  บันทึกการเรียนรู้
}
\lesson{13}
{
\underline{Graphics with Python}
\begin{itemize}
\item Graphic module
\end{itemize}
}
{
\item ทดสอบทบทวนความรู้
\item  บรรยายและยกตัวอย่างการเขียนโปรแกรม
\item  ถามตอบในชั้นเรียน
\item  ปฏิบัติการเขียนโปรแกรม
\item  บันทึกการเรียนรู้
}
\lesson{14}
{
\underline{Data Analytics with Python}
\begin{itemize}
\item Data Analytics
\item Data Visualization
\end{itemize}
}
{
\item ทดสอบทบทวนความรู้
\item  บรรยายและยกตัวอย่างการเขียนโปรแกรม
\item  ถามตอบในชั้นเรียน
\item  ปฏิบัติการเขียนโปรแกรม
\item  บันทึกการเรียนรู้
}
\lesson{15}
{
\underline{Python project 2}
\begin{itemize}
\item ดำเนินการคิดและพัฒนา Project
\item นำเสนอ Project
\end{itemize}
}
{
\item  ปฏิบัติการเขียนโปรแกรม
\item  นำเสนอผลงาน
}


\section*{การจัดการประสบการณ์การเรียนรู้}

\begin{itemize}
\item บรรยายและถ่ายทอดประสบการณ์แก่ผู้เรียน
\item ถามตอบในชั้นเรียน
\item ฝึกปฏิบัติการเขียนโปรแกรม
\item  ฝึกนำเสนอผลงานการเขียนโปรแกรมที่พัฒนาด้วยตนเอง
\item  บันทึกสะท้อนสิ่งที่ได้เรียนรู้
\item  ทดสอบย่อยเพื่อทบทวนความรู้ความเข้าใจในแต่ละบท
\item  สอบปฏิบัติเขียนโปรแกรมเพื่อการบูรณาการความรู้ที่ได้รับทั้งกลางภาคและปลายภาค
\item  ใช้เอกสารประกอบการสอนเพื่อใช้การทบทวนความรู้ที่ได้รับและฝึกทำแบบฝึกหัดท้ายบท
\end{itemize}


\section*{สื่อการเรียนรู้}

\begin{itemize}
\item เอกสารประกอบการสอนรายวิชา 477-201 การเขียนโปรแกรมคอมพิวเตอร์ 
    \begin{itemize}
    \item (หนังสือ E-book แนะนำให้นักศึกษาพิมพ์ออกมาทั้งเล่ม เพื่อความสะดวกในการเรียน)
    \end{itemize}
\item อุปกรณ์คอมพิวเตอร์ที่นักศึกษาต้องจัดหามาเอง
    \begin{itemize}
    \item Notebook/Desktop computer
    \item Webcam, Microphone, Speaker
    \end{itemize}
\item คลิปวิดีโอออนไลน์สื่อการสอนรายวิชา 477-201 การเขียนโปรแกรมคอมพิวเตอร์
    \begin{itemize}
    \item \url{https://www.youtube.com/classstartacademy}
    \end{itemize}
\item ชั้นเรียนออนไลน์รายวิชา 477-201 การเขียนโปรแกรมคอมพิวเตอร์
    \begin{itemize}
    \item ClassStart \url{https://classstart.org}
    \item Microsoft Teams \url{ https://bit.ly/31F32RF}
    \end{itemize}
\item เว็บไซต์
    \begin{itemize}
    \item \url{https://www.python.org/}
    \item \url{https://code.org/}
    \item \url{https://www.tutorialspoint.com/python3/index.htm}
    \item \url{https://www.w3schools.com/python/}
    \end{itemize}
\end{itemize}


\section*{การวัดผลตัดเกรดแบบอิงเกณฑ์}

\begin{center}
\begin{tabu}{c l r r}
 \hline
 ระดับคะแนน & ช่วงคะแนน \\ [0.5ex] 
 \hline
A & 80 - 100 \\
B+  & 75 - 79.99  \\
B  & 70 - 74.99  \\
C+ & 65 - 69.99  \\
C  & 60 - 64.99  \\
D+  & 55 - 59.99  \\
D  & 50 - 54.99  \\
E  & 0 - 49.99 \\
\hline
\end{tabu}
\end{center}

\section*{การประเมินผล}

\begin{itemize}
\item ร้อยละ 20 สอบกลางภาคแบบปฏิบัติการเขียนโปรแกรม
\item ร้อยละ 20 สอบปลายภาคแบบปฏิบัติการเขียนโปรแกรม
\item ร้อยละ 20 แบบฝึกหัดย่อยแบบปฏิบัติการเขียนโปรแกรม
\item ร้อยละ 20 แบบทดสอบย่อยแบบ MCQs
\item ร้อยละ 10 การเข้าเรียนทั้งออนไลน์และออนไซต์ (ถ้ามี)
\item ร้อยละ 10 บันทึกการเรียนรู้
\end{itemize}

\section*{สัปดาห์ที่ทำการประเมินผล}

\begin{center}
\begin{tabu}{c l r r}
 \hline
 สัปดาห์ & การประเมิน \\ [0.5ex] 
 \hline
2 & Quiz, Exercise\\
3 & Quiz, Exercise \\
4 & Quiz, Exercise  \\
5 & Quiz, Exercise  \\
6 & Quiz, Exercise  \\
7 & Project 1  \\
8 & Quiz, Exercise\\
9 & Quiz, Exercise \\
10 & Quiz, Exercise  \\
11 & Quiz, Exercise  \\
12 & Quiz, Exercise  \\
15 & Project 2  \\
\hline
\end{tabu}
\end{center}



\section*{นโยบายการเข้าเรียนออนไลน์}
\definecolor{myblue}{cmyk}{0.80, 0.13, 0.14, 0.04, 1.00}
\begin{tcolorbox}[breakable,enhanced,fonttitle=\bfseries,colback=myblue!05,colframe=myblue]
\begin{itemize}
\item นักศึกษาจะต้องเข้าเรียนทางออนไลน์ทางห้องประชุมออนไลน์ของรายวิชาทุกครั้ง
\item หากนักศึกษามีความจำเป็นต้องขาดเรียนต้องแจ้งให้ผู้สอนทราบก่อนล่วงหน้าโดยเร็วที่สุด
\item หากมีความจำเป็นเร่งด่วน อาทิ ป่วย ให้ติดต่อผู้สอนให้ทราบทันทีทางไลน์ โทรศัพท์ หรือ อีเมล  
\item การขาดเรียนมากกว่าสองครั้งแต่ไม่ได้แจ้งให้ผู้สอนทราบ จะส่งผลให้ถูกปรับโทษ 5 คะแนนต่อการขาดเรียนหลังจากครั้งที่สอง
\end{itemize}
\end{tcolorbox}

\section*{นโยบายประเมินผลการเรียนทางออนไลน์}
\definecolor{myblue}{cmyk}{0.80, 0.13, 0.14, 0.04, 1.00}
\begin{tcolorbox}[breakable,enhanced,fonttitle=\bfseries,colback=myblue!05,colframe=myblue]
\begin{itemize}
\item คำตอบในการประเมินผลการเรียน อาทิ การทำการบ้าน การทำแบบทดสอบ การทำแบบฝึกหัด การทำโครงงาน และงานมอบหมายอื่นๆ จะต้องทำด้วยตัวนักศึกษาเอง (ยกเว้นงานที่ผู้สอนมอบหมายอย่างชัดเจนว่าอนุญาตให้ทำงานร่วมกัน)
\item ในการทำแบบทดสอบออนไลน์ ห้ามนักศึกษาทำการเปิดหนังสือ เอกสาร หรือเว็บไซต์ใดๆ และห้ามทำการหาคำตอบด้วยการประมวลผลโค้ดของโจทย์ด้วยโปรแกรมคอมพิวเตอร์
\item ห้ามนักศึกษาทำเฉลยหรือทำงานแทนผู้อื่น ทั้งในการทำการบ้าน การทำแบบทดสอบ การทำแบบฝึกหัด การทำโครงงาน และงานมอบหมายอื่นๆ (ยกเว้นงานที่ผู้สอนมอบหมายอย่างชัดเจนว่าอนุญาตให้แชร์คำตอบได้)
\item ห้ามนักศึกษาทำกิจกรรมใดๆ อันเป็นการทุจริตต่อการประเมินผล  ทั้งในการทำการบ้าน การทำแบบทดสอบ การทำแบบฝึกหัด การทำโครงงาน และงานมอบหมายอื่นๆ 
\item การขาดสอบ การไม่ส่งงานตามเวลาที่กำหนด จะไม่ได้คะแนนของงานชิ้นนั้น (ยกเว้นมีหลักฐานของสาเหตุการขาดสอบหรือการไม่ได้ทำงานส่งอย่างชัดเจน อาทิ ใบรับรองแพทย์การลาป่วย)
\item หากผู้สอนพบการทุจริต นักศึกษาจะถูกปรับตกและได้รับเกรด E ในรายวิชานี้
\end{itemize}
\end{tcolorbox}

\section*{การเตรียมความพร้อมสำหรับสถานการณ์ฉุกเฉินอันเนื่องจากโควิด-19}
\definecolor{myblue}{cmyk}{0.80, 0.13, 0.14, 0.04, 1.00}
\begin{tcolorbox}[breakable,enhanced,fonttitle=\bfseries,colback=myblue!05,colframe=myblue]
\begin{itemize}
\item นักศึกษาจะต้องตรวจสอบข้อความแจ้งเตือนจากผู้สอนอย่างสม่ำเสมอทั้งทางไลน์กลุ่ม ระบบสนทนา ClassStart และ Microsoft Teams ของรายวิชา
\item หากผู้สอนพบเห็นพฤติกรรมของนักศึกษาในการเพิ่มความเสี่ยงต่อการติดโรคโควิด-19 ผู้สอนมีสิทธิ์กำหนดการสอนเป็นการสอนออนไลน์ทั้งหมด
\item หากนักศึกษามีความเครียดหรือความวิตกกังวลใดๆ อันเนื่องมาจากสถานการณ์โควิด-19 ให้แจ้งผู้สอน อาจารย์ที่ปรึกษา หรือหน่วยกิจการนักศึกษาของคณะได้ทันที
\end{itemize}
\end{tcolorbox}

\emph{\textbf{หมายเหตุ: เอกสารแผนการสอนอาจมีการเปลี่ยนแปลง}}

%\newcommand{\weeklyevaluation}[3]{
%\begin{tcolorbox}[breakable,enhanced,fonttitle=\bfseries,title=สัปดาห์ที่ {#1},colback=myblue!05,colframe=myblue]
%	\vspace{0.5cm}
%	\begin{description}
%	\item[วัตถุประสงค์การประเมิน] \hfill \\
%	{#2}
%	\item[วิธีการประเมิน] \hfill
%	{#3}
%	\end{description}
%	\vspace{1cm}
%\end{tcolorbox}
%\vspace{1.5cm}
%% \newpage
%}
%
%\weeklyevaluation{3}{
%	เมื่อฟังการบรรยาย ถามตอบในชั้นเรียน และฝึกปฏิบัติการเขียนโปรแกรม Python เกี่ยวกับเนื้อหาบทที่1-2 ความรู้เบื้องต้นเกี่ยวกับ Python และ ส่วนประกอบต่างๆ ของภาษา Python แล้วผู้เรียนสามารถ 
%	\begin{itemize}
%	\item อธิบายหลักการหน้าที่ของโปรแกรมคอมพิวเตอร์ได้
%	\item อธิบายการทำงานของ Python ได้
%	\item วิเคราะห์อัลกอริทึมและเขียนผังงานได้
%	\item ติดตั้งโปรแกรมจัดการ Python Runtime และ IDE ได้
%	\item ใช้ตัวแปรและเครื่องหมายคำนวณทางคณิตศาสตร์ในการเขียนโปรแกรมได้
%	\item ใช้คำสั่งพื้นฐาน print และ input ในการเขียนโปรแกรมได้
%	\item เขียนโปรแกรมภาษา Python ที่มีโครงสร้างตามลำดับได้
%	\end{itemize}
%}{
%	\begin{itemize}
%	\item แบบฝึกหัดย่อยแบบปฏิบัติการเขียนโปรแกรม
%	\item แบบทดสอบย่อยแบบ MCQs
%	\end{itemize}
%}
%
%\weeklyevaluation{5}{
%	 เมื่อฟังการบรรยาย ถามตอบในชั้นเรียน และฝึกปฏิบัติการเขียนโปรแกรมภาษา Python เกี่ยวกับเนื้อหาบทที่ 3 ประโยคเงื่อนไขในภาษา Python แล้วผู้เรียนสามารถ
%	\begin{itemize}
%	\item อธิบายหลักการของ Boolean Expression ได้
%	\item เขียนโปรแกรม Python ที่มีโครงสร้างทางเลือกโดยมีเงื่อนไขได้
%	\end{itemize}
%}{
%	\begin{itemize}
%	\item แบบฝึกหัดย่อยแบบปฏิบัติการเขียนโปรแกรม
%	\item แบบทดสอบย่อยแบบ MCQs
%	\end{itemize}
%}
%
%\weeklyevaluation{7}{
%	 เมื่อฟังการบรรยาย ถามตอบในชั้นเรียน และฝึกปฏิบัติการเขียนโปรแกรม Python เกี่ยวกับเนื้อหาบทที่ 4 การเขียนฟังก์ชันในภาษา Python แล้วผู้เรียนสามารถ
%	\begin{itemize}
%	\item อธิบายหลักการของฟังก์ชันหรือโปรแกรมย่อย
%	\item เข้าใจวิธีการแบ่งโปรแกรมใหญ่เป็นโปรแกรมย่อยและเรียกใช้โปรแกรมย่อยได้
%	\item เขียนฟังก์ชันในภาษา Python ได้
%	\end{itemize}
%}{
%	\begin{itemize}
%	\item แบบฝึกหัดย่อยแบบปฏิบัติการเขียนโปรแกรม
%	\item แบบทดสอบย่อยแบบ MCQs
%	\end{itemize}
%}
%
%\weeklyevaluation{8}{
%	 เมื่อฟังการบรรยาย ถามตอบในชั้นเรียน และฝึกปฏิบัติการเขียนโปรแกรม Python เกี่ยวกับเนื้อหาบทที่ 5 การใช้ประโยคสั่งทำงานซ้ำในภาษา Python แล้วผู้เรียนสามารถ
%	\begin{itemize}
%	\item เข้าใจกระบวนการทำงานแบบวนซ้ำ
%	\item เขียนโปรแกรมภาษา Python โดยใช้โครงสร้างการทำวนซ้ำแบบ \pyinline{for} และ \pyinline{while} ได้
%	\item เขียนโปรแกรมภาษา Python โดยใช้ฟังก์ชัน \pyinline{range()} ได้
%	\item เขียนโปรแกรมภาษา Python สร้างฟังก์ชันที่เรียกตัวเองได้
%	\end{itemize}
%}{
%	\begin{itemize}
%	\item แบบฝึกหัดย่อยแบบปฏิบัติการเขียนโปรแกรม
%	\item แบบทดสอบย่อยแบบ MCQs
%	\end{itemize}
%}
%
%\weeklyevaluation{9 (สัปดาห์สอบกลางภาคเชิงปฎิบัติการ)}{
%	เพื่อประเมินความรู้ความเข้าใจทั้งหมดของเนื้อหาบทที่ 1-5
%}{
%	\\
%	สอบปฏิบัติการเขียนโปรแกรม
%}
%
%\weeklyevaluation{10}{
%	 เมื่อฟังการบรรยาย ถามตอบในชั้นเรียน และฝึกปฏิบัติการเขียนโปรแกรม Python เกี่ยวกับเนื้อหาบทที่ 6 ชนิดข้อมูล String แล้วผู้เรียนสามารถ
%	\begin{itemize}
%	\item อธิบายหลักการใช้และจัดการกับ String และตัวชี้
%	\item อธิบายโครงสร้างข้อมูลที่เปลี่ยนแปลงไม่ได้ได้
%	\item เขียนโปรแกรมภาษา Python เพื่อจัดการกับ String ได้
%	\item อธิบายและใช้เมธอดของ String ได้
%	\end{itemize}
%}{
%	\begin{itemize}
%	\item แบบฝึกหัดย่อยแบบปฏิบัติการเขียนโปรแกรม
%	\item แบบทดสอบย่อยแบบ MCQs
%	\end{itemize}
%}
%
%\weeklyevaluation{11}{
%	 เมื่อฟังการบรรยาย ถามตอบในชั้นเรียน และฝึกปฏิบัติการเขียนโปรแกรม Python เกี่ยวกับเนื้อหาบทที่ 7 ชนิดข้อมูล List แล้วผู้เรียนสามารถ
%	\begin{itemize}
%	\item อธิบายหลักการใช้และจัดการกับ List และตัวชี้
%	\item อธิบายโครงสร้างข้อมูลที่เปลี่ยนแปลงได้ได้
%	\item เขียนโปรแกรมภาษา Python เพื่อจัดการกับ List ได้
%	\item อธิบายและใช้เมธอดของ List ได้
%	\end{itemize}
%}{
%	\begin{itemize}
%	\item แบบฝึกหัดย่อยแบบปฏิบัติการเขียนโปรแกรม
%	\item แบบทดสอบย่อยแบบ MCQs
%	\end{itemize}
%}
%
%\weeklyevaluation{12}{
%	 เมื่อฟังการบรรยาย ถามตอบในชั้นเรียน และฝึกปฏิบัติการเขียนโปรแกรม Python เกี่ยวกับเนื้อหาบทที่ 8 ชนิดข้อมูล Dictionary แล้วผู้เรียนสามารถ
%	\begin{itemize}
%	\item อธิบายหลักการใช้และจัดการกับ Dictionary และตัวชี้
%	\item เขียนโปรแกรมภาษา Python เพื่อจัดการกับ Dictionary ได้
%	\item อธิบายและใช้เมธอดของ Dictionary ได้
%	\end{itemize}
%}{
%	\begin{itemize}
%	\item แบบฝึกหัดย่อยแบบปฏิบัติการเขียนโปรแกรม
%	\item แบบทดสอบย่อยแบบ MCQs
%	\end{itemize}
%}
%
%\weeklyevaluation{13}{
%	 เมื่อฟังการบรรยาย ถามตอบในชั้นเรียน และฝึกปฏิบัติการเขียนโปรแกรม Python เกี่ยวกับเนื้อหาบทที่ 9 ชนิดข้อมูล Tuple แล้วผู้เรียนสามารถ
%	\begin{itemize}
%	\item อธิบายหลักการใช้และจัดการกับ Tuple และตัวชี้
%	\item เขียนโปรแกรมภาษา Python เพื่อจัดการกับ Tuple ได้
%	\item อธิบายและใช้เมธอดของ Tuple ได้
%	\end{itemize}
%}{
%	\begin{itemize}
%	\item แบบฝึกหัดย่อยแบบปฏิบัติการเขียนโปรแกรม
%	\item แบบทดสอบย่อยแบบ MCQs
%	\end{itemize}
%}
%
%\weeklyevaluation{14}{
%	 เมื่อฟังการบรรยาย ถามตอบในชั้นเรียน และฝึกปฏิบัติการเขียนโปรแกรม Python เกี่ยวกับเนื้อหาบทที่ 10 การจัดการ Files แล้วผู้เรียนสามารถ
%	\begin{itemize}
%	\item อธิบายหลักการใช้และจัดการกับ Files ใน Python ได้
%	\item เขียนโปรแกรม Python เพื่อจัดการกับ Files ได้
%	\item เขียนโปรแกรม Python เพื่อจัดการกับฐานข้อมูลแบบ Key-Value ได้
%	\end{itemize}
%}{
%	\begin{itemize}
%	\item แบบฝึกหัดย่อยแบบปฏิบัติการเขียนโปรแกรม
%	\item แบบทดสอบย่อยแบบ MCQs
%	\end{itemize}
%}
%
%\weeklyevaluation{15}{
%	 เมื่อฟังการบรรยาย ถามตอบในชั้นเรียน และฝึกปฏิบัติการเขียนโปรแกรม Python เกี่ยวกับเนื้อหาบทที่ 11 Object-oriented programming (OOP) แล้วผู้เรียนสามารถ
%	\begin{itemize}
%	\item อธิบายหลักการพื้นฐานเขียนโปรแกรม Python แบบ OOP ได้
%	\item อธิบายหลักการทำงานของคลาสและออบเจ็กต์
%	\item อธิบายหลักการทำงานแบบ Inheritance และ Polymorphism ได้
%	\item เขียนโปรแกรม Python แบบ OOP พื้นฐานได้
%	\end{itemize}
%}{
%	\begin{itemize}
%	\item แบบฝึกหัดย่อยแบบปฏิบัติการเขียนโปรแกรม
%	\item แบบทดสอบย่อยแบบ MCQs
%	\end{itemize}
%}
%
%\weeklyevaluation{16 (สัปดาห์สอบปลายภาคเชิงปฎิบัติการ)}{
%	เพื่อประเมินความรู้ความเข้าใจทั้งหมดของเนื้อหาบทที่ 6-11
%}{
%	\\ สอบปฏิบัติการเขียนโปรแกรม
%}
%
%\newpage
%
%
%
%
%\vspace{7mm}
%\noindent
%\hangindent=9cm
%\hangafter=0
%ลงชื่อ\\
%\vspace{10mm}\\
%\textbf{ดร.จันทวรรณ ปิยะวัฒน์}\\
%ผู้สอน



