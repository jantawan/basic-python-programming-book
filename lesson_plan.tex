\chapter{แผนการสอน}
\section*{คำอธิบายรายวิชาและวัตถุประสงค์ที่ระบุไว้ในหลักสูตร}

\begin{tcolorbox}[breakable,enhanced,fonttitle=\bfseries]
แนวความคิดเรื่องการเขียนโปรแกรม ขั้นตอนวิธีในการแก้ไขปัญหา การสร้างคำสั่งสำหรับเขียนขั้นตอนวิธีการ เขียนผังงาน นิพจน์ คำสั่งในการเขียนโปรแกรม หลักไวยากรณ์ของภาษาโปรแกรมระดับสูง การเขียนโปรแกรมสมัยใหม่ การทดสอบ การแก้ไขโปรแกรม การติดตั้ง และการเขียนเอกสารประกอบโปรแกรม

Concept of programming, algorithm to solve the problem, flowchart, expression and instruction, high-level language syntax, modern programming, testing, debugging, installation and software documentation
\end{tcolorbox}

\section*{วัตถุประสงค์ของวิชา}

\begin{tcolorbox}[breakable,enhanced,fonttitle=\bfseries]
มีจุดมุ่งหมายให้นักศึกษาได้มีพื้นฐานความรู้ความเข้าใจในหลักการเขียนโปรแกรมคอมพิวเตอร์เบื้องต้นด้วยภาษา Python ส่วนประกอบต่างๆ ของโปรแกรมคอมพิวเตอร์และภาษา Python สามารถเขียนโปรแกรมคอมพิวเตอร์อย่างง่ายด้วยภาษา Python ได้ตามการวิเคราะห์และออกแบบขั้นตอนการทำงานของโปรแกรมอย่างมีระบบ และมีความรู้ความเข้าใจในการเขียนโปรแกรมแบบมีเงื่อนไขเพื่อการตัดสินใจ การเขียนคำสั่งเพื่อการทำงานซ้ำ และโมดูลส่วนเสริมต่างๆ ของโปรแกรมภาษา Python เพื่อเรียนรู้เรื่องการเขียนโปรแกรมแบบฟังก์ชันและการเขียนโปรแกรมเชิงวัตถุได้
\end{tcolorbox}
\vspace{1.5cm}

\section*{เนื้อหาวิชา}

% \lesson command definition

\newcommand{\lesson}[3]{
% \subsection*{สัปดาห์ที่ {#1}}
\begin{tcolorbox}[breakable,enhanced,fonttitle=\bfseries,title=สัปดาห์ที่ {#1}]
\begin{description}

\item[ผู้สอน] จันทวรรณ ปิยะวัฒน์
\item[จำนวนชั่วโมงบรรยาย] 2
\item[จำนวนชั่วโมงปฎิบัติ] 2
\item[หัวข้อ/รายละเอียด] \hfill \\
{#2}
\item[กิจกรรมการเรียนการสอน/สื่อที่ใช้] \hfill
\begin{itemize}[leftmargin=0pt]
{#3}
\end{itemize}

\end{description}
\end{tcolorbox}
\newpage
}

% --

\lesson{1}
{
\underline{เค้าโครงวิชา}

\begin{itemize}
\item วัตถุประสงค์รายวิชา
\item รายละเอียดเนื้อหาวิชา
\item การวัดผลและการประเมินผล
\item เงื่อนไขและข้อตกลงอื่น
\item วิธีการเรียนการสอน
\item เว็บไซต์และหนังสืออ่านประกอบ
\end{itemize}

\underline{ระบบจัดการการเรียนรู้ (ClassStart.org)}
\begin{itemize}
\item ระบบในภาพรวม
\item การสมัครสมาชิก
\item การเข้าห้องเรียนออนไลน์ของรายวิชา
\item การใช้งานระบบ
\item การเข้าอ่านเอกสารการสอนและคลิป
\item การส่งแบบฝึกหัดทางออนไลน์
\item การทำข้อสอบออนไลน์
\item การตรวจสอบคะแนนเก็บ
\item การบันทึกการเรียนรู้ (Reflections)
\item การสื่อสารออนไลน์
\end{itemize}

\underline{เว็บไซต์ Code.org}
\begin{itemize}
\item การสมัครสมาชิก
\item ฝึกการเขียนโปรแกรมง่าย ๆ (Game-based Learning) แบบ Block-based Programming
\end{itemize}
}
{
\item บรรยาย
\item ปฎิบัติการใช้ระบบ ClassStart.org
\item ปฎิบัติการเขียนโปรแกรมทางออนไลน์ที่ Code.org
}

\lesson{2}
{
\underline{บทที่ 1 ความรู้เบื้องต้นเกี่ยวกับ Python}
\begin{itemize}
\item Python คืออะไร
\item  Python ทำงานอย่างไร
\item  อัลกอริทึมและผังงาน
\item  การติดตั้งโปรแกรม Python Runtime
\end{itemize}
}
{
\item บรรยายและยกตัวอย่างการเขียนโปรแกรม
\item  ถามตอบในชั้นเรียน
\item  ฝึกการเขียนผังงาน
\item  ปฏิบัติการติดตั้ง Python Runtime
\item  ปฏิบัติการเขียนโปรแกรม
\item  บันทึกการเรียนรู้
}
\lesson{3}
{
\underline{บทที่ 2 ส่วนประกอบของ Python}
\begin{itemize}
\item ตัวแปร
\item ประเภทของข้อมูล
\item การคำนวณ
\item Expressions และ Statements
\item Comments
\item Source Code
\item คำสั่ง print()
\item คำสั่ง input()
\end{itemize}
}
{
\item ทดสอบทบทวนความรู้
\item  บรรยายและยกตัวอย่างการเขียนโปรแกรม
\item  ถามตอบในชั้นเรียน
\item  ปฏิบัติการเขียนโปรแกรม
\item  บันทึกการเรียนรู้
}
\lesson{4}
{
\underline{บทที่ 3 ประโยคเงื่อนไข}
\begin{itemize}
\item Boolean Expressions
\item การใช้ if กับ else
\end{itemize}
}
{
\item ทดสอบทบทวนความรู้
\item  บรรยายและยกตัวอย่างการเขียนโปรแกรม
\item  ถามตอบในชั้นเรียน
\item  ปฏิบัติการเขียนโปรแกรม
\item  บันทึกการเรียนรู้
}
\lesson{5}
{
\underline{บทที่ 3 ประโยคเงื่อนไข (ต่อ)}
\begin{itemize}
\item Chained Expressions
\item  Nested Expressions
\end{itemize}
}
{
\item ทดสอบทบทวนความรู้
\item  บรรยายและยกตัวอย่างการเขียนโปรแกรม
\item  ถามตอบในชั้นเรียน
\item  ปฏิบัติการเขียนโปรแกรม
\item  บันทึกการเรียนรู้
}
\lesson{6}
{
\underline{บทที่ 4 การเขียนและใช้งานฟังก์ชัน}
\begin{itemize}
\item การเรียกใช้ฟังก์ชัน
\item การเรียกใช้โมดูล
\item ฟังก์ชันซ้อน
\end{itemize}
}
{
\item  บรรยายและยกตัวอย่างการเขียนโปรแกรม
\item  ถามตอบในชั้นเรียน
\item  ปฏิบัติการเขียนโปรแกรม
\item  บันทึกการเรียนรู้
}
\lesson{7}
{
\underline{บทที่ 4 การเขียนและใช้งานฟังก์ชัน (ต่อ)}
\begin{itemize}
\item การสร้างฟังก์ชัน
\item การคืนค่าของฟังก์ชัน
\item การเขียนโปรแกรมเชิงฟังก์ชัน
\item การเขียนคำอธิบายโปรแกรม
\end{itemize}
\underline{ทบทวนเนื้อหาก่อนสอบกลางภาค}
}
{
\item ทดสอบทบทวนความรู้
\item  บรรยายและยกตัวอย่างการเขียนโปรแกรม
\item  ถามตอบในชั้นเรียน
\item  ปฏิบัติการเขียนโปรแกรม
\item  บันทึกการเรียนรู้
}
\lesson{8}
{
\underline{บทที่ 5 การใช้ประโยคสั่งทำงานวนซ้ำ}
\begin{itemize}
\item ฟังก์ชัน \texttt{range()}
\item คำสั่ง \texttt{for}
\end{itemize}

}
{
\item  บรรยายและยกตัวอย่างการเขียนโปรแกรม
\item  ถามตอบในชั้นเรียน
\item  ปฏิบัติการเขียนโปรแกรม
\item  บันทึกการเรียนรู้
}
\lesson{9}
{
\underline{บทที่ 5 การใช้ประโยคสั่งทำงานวนซ้ำ (ต่อ)}
\begin{itemize}
\item คำสั่ง \texttt{while}
\item คำสั่ง \texttt{break}
\item ฟังก์ชันที่เรียกตัวเอง
\end{itemize}
}
{
\item ทดสอบทบทวนความรู้
\item  บรรยายและยกตัวอย่างการเขียนโปรแกรม
\item  ถามตอบในชั้นเรียน
\item  ปฏิบัติการเขียนโปรแกรม
\item  บันทึกการเรียนรู้
}
\lesson{10}
{
\underline{บทที่ 6 การใช้งาน String}
\begin{itemize}
\item ฟังก์ชัน \texttt{len()}
\item การเดินทางตามตัวชี้ของ String
\item การตัดคำใน String
\item โครงสร้างข้อมูลที่เปลี่ยนแปลงไม่ได้
\item การค้นหาตัวอักษรใน String
\item String Methods
\item การใช้ in
\item การเปรียบเทียบ String
\item การจัดวางรูปแบบของ String

\end{itemize}

}
{
\item ทดสอบทบทวนความรู้
\item  บรรยายและยกตัวอย่างการเขียนโปรแกรม
\item  ถามตอบในชั้นเรียน
\item  ปฏิบัติการเขียนโปรแกรม
\item  บันทึกการเรียนรู้
}
\lesson{11}
{
\underline{บทที่ 7 ลิสต์ (List)}
\begin{itemize}
\item การเข้าถึงค่าในลิสต์
\item การแบ่งข้อมูลในลิสต์
\item การใช้ in กับลิสต์
\item การเดินทางในลิสต์
\item ตัวเนินการของลิสต์
\item List Methods
\item Map, reduce, and filter
\item Lists กับ String
\item Objects กับ values
\end{itemize}
}
{
\item ทดสอบทบทวนความรู้
\item  บรรยายและยกตัวอย่างการเขียนโปรแกรม
\item  ถามตอบในชั้นเรียน
\item  ปฏิบัติการเขียนโปรแกรม
\item  บันทึกการเรียนรู้
}
\lesson{12}
{
\underline{บทที่ 8 ดิกชันนารี (Dictionary)}
\begin{itemize}
\item การอ่านค่าใน Dictionary
\item การหาค่าของ Key ใน Dictionary
\item Dictionary and List
\item ฟังก์ชันที่รับ Parameters ได้ไม่จำกัด
\end{itemize}
}
{
\item ทดสอบทบทวนความรู้
\item  บรรยายและยกตัวอย่างการเขียนโปรแกรม
\item  ถามตอบในชั้นเรียน
\item  ปฏิบัติการเขียนโปรแกรม
\item  บันทึกการเรียนรู้
}
\lesson{13}
{
\underline{บทที่ 9 ทูเบิล (Tuple)}
\begin{itemize}
\item ความหมายของ Tuple
\item การสลับค่าของ Tuple
\item การเก็บค่าการดำเนินการใน Tuple
\item ฟังก์ชัน list()
\item Dictionary และ Tuple
\end{itemize}
}
{
\item ทดสอบทบทวนความรู้
\item  บรรยายและยกตัวอย่างการเขียนโปรแกรม
\item  ถามตอบในชั้นเรียน
\item  ปฏิบัติการเขียนโปรแกรม
\item  บันทึกการเรียนรู้
}
\lesson{14}
{
\underline{บทที่ 10 การจัดการไฟล์ (Files)}
\begin{itemize}
\item การทำงานกับ Directories
\item การเปิดไฟล์
\item การอ่านไฟล์
\item การจัดการข้อผิดพลาด
\item ฐานข้อมูลแบบ Key-Value
\item การเรียกใช้โปรแกรมอื่น
\end{itemize}
}
{
\item ทดสอบทบทวนความรู้
\item  บรรยายและยกตัวอย่างการเขียนโปรแกรม
\item  ถามตอบในชั้นเรียน
\item  ปฏิบัติการเขียนโปรแกรม
\item  บันทึกการเรียนรู้
}
\lesson{15}
{
\underline{บทที่ 11 Object-Oriented Programming}
\begin{itemize}
\item คลาสและออบเจ็กต์
\item การสร้างคลาส
\item การสร้างออบเจ็กต์
\item ฟังก์ชัน \texttt{\_\_init\_\_()}
\item การสร้างเมธอดของออบเจ็กต์
\item การแก้ไขค่าแอตทริบิวต์ของออบเจ็กต์
\item การลบแอตทริบิวต์ของออบเจ็กต์
\item การลบออบเจ็กต์
\item การสืบทอดคลาส
\end{itemize}

\underline{ทบทวนเนื้อหาก่อนสอบปลายภาค}
}
{
\item ทดสอบทบทวนความรู้
\item  บรรยายและยกตัวอย่างการเขียนโปรแกรม
\item  ถามตอบในชั้นเรียน
\item  ปฏิบัติการเขียนโปรแกรม
\item  บันทึกการเรียนรู้
}