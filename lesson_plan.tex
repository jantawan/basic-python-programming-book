\chapter{แผนการสอน}
\section*{คำอธิบายรายวิชาและวัตถุประสงค์ที่ระบุไว้ในหลักสูตร}

\begin{tcolorbox}[breakable,enhanced,fonttitle=\bfseries]
แนวความคิดเรื่องการเขียนโปรแกรม ขั้นตอนวิธีในการแก้ไขปัญหา การสร้างคำสั่งสำหรับเขียนขั้นตอนวิธีการ เขียนผังงาน นิพจน์ คำสั่งในการเขียนโปรแกรม หลักไวยากรณ์ของภาษาโปรแกรมระดับสูง การเขียนโปรแกรมสมัยใหม่ การทดสอบ การแก้ไขโปรแกรม การติดตั้ง และการเขียนเอกสารประกอบโปรแกรม

Concept of programming, Algorithm to solve the problem, Flowchart, Expression and instruction, High-level language syntax, Modern programming, Testing, Debugging, Installation and software documentation
\end{tcolorbox}

\section*{วัตถุประสงค์ของวิชา}

\begin{tcolorbox}[breakable,enhanced,fonttitle=\bfseries]
มีจุดมุ่งหมายให้นักศึกษาได้มีพื้นฐานความรู้ความเข้าใจในหลักการเขียนโปรแกรมคอมพิวเตอร์เบื้องต้นด้วยภาษา Python ส่วนประกอบต่างๆ ของโปรแกรมคอมพิวเตอร์และภาษา Python สามารถเขียนโปรแกรมคอมพิวเตอร์อย่างง่ายด้วยภาษา Python ได้ตามการวิเคราะห์และออกแบบขั้นตอนการทำงานของโปรแกรมอย่างมีระบบ และมีความรู้ความเข้าใจในการเขียนโปรแกรมแบบมีเงื่อนไขเพื่อการตัดสินใจ การเขียนคำสั่งเพื่อการทำงานซ้ำ และโมดูลส่วนเสริมต่างๆ ของโปรแกรมภาษา Python เพื่อเรียนรู้เรื่องการเขียนโปรแกรมแบบฟังก์ชันและการเขียนโปรแกรมเชิงวัตถุได้
\end{tcolorbox}
\vspace{1.5cm}

\section*{เนื้อหาวิชา}

% \lesson command definition

\newcommand{\lesson}[3]{
% \subsection*{สัปดาห์ที่ {#1}}
\begin{tcolorbox}[breakable,enhanced,fonttitle=\bfseries,title=สัปดาห์ที่ {#1}]
\begin{description}

\item[ผู้สอน] จันทวรรณ ปิยะวัฒน์
\item[จำนวนชั่วโมงบรรยาย] 2
\item[จำนวนชั่วโมงปฎิบัติ] 2
\item[หัวข้อ/รายละเอียด] \hfill \\
{#2}
\item[กิจกรรมการเรียนการสอน/สื่อที่ใช้] \hfill
\begin{itemize}[leftmargin=0pt]
{#3}
\end{itemize}

\end{description}
\end{tcolorbox}
\newpage
}

% --

\lesson{1}
{
\underline{เค้าโครงวิชา}

\begin{itemize}
\item วัตถุประสงค์รายวิชา
\item รายละเอียดเนื้อหาวิชา
\item การวัดผลและการประเมินผล
\item เงื่อนไขและข้อตกลงอื่น
\item วิธีการเรียนการสอน
\item เว็บไซต์และหนังสืออ่านประกอบ
\end{itemize}

\underline{ระบบจัดการการเรียนรู้ (ClassStart.org)}
\begin{itemize}
\item ระบบในภาพรวม
\item การสมัครสมาชิก
\item การเข้าห้องเรียนออนไลน์ของรายวิชา
\item การใช้งานระบบ
\item การเข้าอ่านเอกสารการสอนและคลิป
\item การส่งแบบฝึกหัดทางออนไลน์
\item การทำข้อสอบออนไลน์
\item การตรวจสอบคะแนนเก็บ
\item การบันทึกการเรียนรู้ (Reflections)
\item การสื่อสารออนไลน์
\end{itemize}

\underline{เว็บไซต์ Code.org}
\begin{itemize}
\item การสมัครสมาชิก
\item ฝึกการเขียนโปรแกรมง่ายๆ (Game-based Learning) แบบ Block-based Programming
\end{itemize}
}
{
\item บรรยาย
\item ปฎิบัติการใช้ระบบ ClassStart.org
\item ปฎิบัติการเขียนโปรแกรมทางออนไลน์ที่ Code.org
}

\lesson{2}
{
\underline{บทที่ 1 ความรู้เบื้องต้นเกี่ยวกับ Python}
\begin{itemize}
\item Python คืออะไร
\item  Python ทำงานอย่างไร
\item  อัลกอริทึมและผังงาน
\item  การติดตั้งโปรแกรม Python Runtime
\end{itemize}
}
{
\item บรรยายและยกตัวอย่างการเขียนโปรแกรม
\item  ถามตอบในชั้นเรียน
\item  ฝึกการเขียนผังงาน
\item  ปฏิบัติการติดตั้ง Python Runtime
\item  ปฏิบัติการเขียนโปรแกรม
\item  บันทึกการเรียนรู้
}
\lesson{3}
{
\underline{บทที่ 2 ส่วนประกอบของ Python}
\begin{itemize}
\item ตัวแปร
\item ประเภทของข้อมูล
\item การคำนวณ
\item Expressions และ Statements
\item Comments
\item Source Code
\item คำสั่ง  \texttt{print()}
\item คำสั่ง  \texttt{input()}
\end{itemize}
}
{
\item ทดสอบทบทวนความรู้
\item  บรรยายและยกตัวอย่างการเขียนโปรแกรม
\item  ถามตอบในชั้นเรียน
\item  ปฏิบัติการเขียนโปรแกรม
\item  บันทึกการเรียนรู้
}
\lesson{4}
{
\underline{บทที่ 3 ประโยคเงื่อนไข}
\begin{itemize}
\item Boolean Expressions
	\item การใช้  \texttt{if},  \texttt{elif} และ  \texttt{else}
\end{itemize}
}
{
\item ทดสอบทบทวนความรู้
\item  บรรยายและยกตัวอย่างการเขียนโปรแกรม
\item  ถามตอบในชั้นเรียน
\item  ปฏิบัติการเขียนโปรแกรม
\item  บันทึกการเรียนรู้
}
\lesson{5}
{
\underline{บทที่ 3 ประโยคเงื่อนไข (ต่อ)}
\begin{itemize}
\item Chained Expressions
\item  Nested Expressions
\end{itemize}
}
{
\item ทดสอบทบทวนความรู้
\item  บรรยายและยกตัวอย่างการเขียนโปรแกรม
\item  ถามตอบในชั้นเรียน
\item  ปฏิบัติการเขียนโปรแกรม
\item  บันทึกการเรียนรู้
}
\lesson{6}
{
\underline{บทที่ 4 การเขียนและใช้งานฟังก์ชัน}
\begin{itemize}
\item การเรียกใช้ฟังก์ชัน
\item การเรียกใช้โมดูล
\item ฟังก์ชันซ้อน
\end{itemize}
}
{
\item  บรรยายและยกตัวอย่างการเขียนโปรแกรม
\item  ถามตอบในชั้นเรียน
\item  ปฏิบัติการเขียนโปรแกรม
\item  บันทึกการเรียนรู้
}
\lesson{7}
{
\underline{บทที่ 4 การเขียนและใช้งานฟังก์ชัน (ต่อ)}
\begin{itemize}
\item การสร้างฟังก์ชัน
\item การคืนค่าของฟังก์ชัน
\item การเขียนโปรแกรมเชิงฟังก์ชัน
\item การเขียนคำอธิบายโปรแกรม
\end{itemize}
\underline{ทบทวนเนื้อหาก่อนสอบกลางภาค}
}
{
\item ทดสอบทบทวนความรู้
\item  บรรยายและยกตัวอย่างการเขียนโปรแกรม
\item  ถามตอบในชั้นเรียน
\item  ปฏิบัติการเขียนโปรแกรม
\item  บันทึกการเรียนรู้
}
\lesson{8}
{
\underline{บทที่ 5 การใช้ประโยคสั่งทำงานวนซ้ำ}
\begin{itemize}
\item ฟังก์ชัน \texttt{range()}
\item คำสั่ง \texttt{for}
\end{itemize}

}
{
\item  บรรยายและยกตัวอย่างการเขียนโปรแกรม
\item  ถามตอบในชั้นเรียน
\item  ปฏิบัติการเขียนโปรแกรม
\item  บันทึกการเรียนรู้
}
\lesson{9}
{
\underline{บทที่ 5 การใช้ประโยคสั่งทำงานวนซ้ำ (ต่อ)}
\begin{itemize}
\item คำสั่ง \texttt{while}
\item คำสั่ง \texttt{break}
\item ฟังก์ชันที่เรียกตัวเอง
\end{itemize}
}
{
\item ทดสอบทบทวนความรู้
\item  บรรยายและยกตัวอย่างการเขียนโปรแกรม
\item  ถามตอบในชั้นเรียน
\item  ปฏิบัติการเขียนโปรแกรม
\item  บันทึกการเรียนรู้
}
\lesson{10}
{
\underline{บทที่ 6 การใช้งาน String}
\begin{itemize}
\item ฟังก์ชัน \texttt{len()}
\item การเดินทางตามตัวชี้ของ String
\item การตัดคำใน String
\item โครงสร้างข้อมูลที่เปลี่ยนแปลงไม่ได้
\item การค้นหาตัวอักษรใน String
\item String Methods
\item การใช้  \texttt{in}
\item การเปรียบเทียบ String
\item การจัดวางรูปแบบของ String

\end{itemize}

}
{
\item ทดสอบทบทวนความรู้
\item  บรรยายและยกตัวอย่างการเขียนโปรแกรม
\item  ถามตอบในชั้นเรียน
\item  ปฏิบัติการเขียนโปรแกรม
\item  บันทึกการเรียนรู้
}
\lesson{11}
{
\underline{บทที่ 7 ลิสต์ (List)}
\begin{itemize}
\item การเข้าถึงค่าในลิสต์
\item การแบ่งข้อมูลในลิสต์
\item การใช้ in กับลิสต์
\item การเดินทางในลิสต์
\item ตัวเนินการของลิสต์
\item List Methods
\item Map, reduce, and filter
\item Lists กับ String
\item Objects กับ values
\end{itemize}
}
{
\item ทดสอบทบทวนความรู้
\item  บรรยายและยกตัวอย่างการเขียนโปรแกรม
\item  ถามตอบในชั้นเรียน
\item  ปฏิบัติการเขียนโปรแกรม
\item  บันทึกการเรียนรู้
}
\lesson{12}
{
\underline{บทที่ 8 ดิกชันนารี (Dictionary)}
\begin{itemize}
\item การอ่านค่าใน Dictionary
\item การหาค่าของ Key ใน Dictionary
\item Dictionary and List
\item ฟังก์ชันที่รับ Parameters ได้ไม่จำกัด
\end{itemize}
}
{
\item ทดสอบทบทวนความรู้
\item  บรรยายและยกตัวอย่างการเขียนโปรแกรม
\item  ถามตอบในชั้นเรียน
\item  ปฏิบัติการเขียนโปรแกรม
\item  บันทึกการเรียนรู้
}
\lesson{13}
{
\underline{บทที่ 9 ทูเบิล (Tuple)}
\begin{itemize}
\item ความหมายของ Tuple
\item การสลับค่าของ Tuple
\item การเก็บค่าการดำเนินการใน Tuple
\item ฟังก์ชัน  \texttt{list()}
\item Dictionary และ Tuple
\end{itemize}
}
{
\item ทดสอบทบทวนความรู้
\item  บรรยายและยกตัวอย่างการเขียนโปรแกรม
\item  ถามตอบในชั้นเรียน
\item  ปฏิบัติการเขียนโปรแกรม
\item  บันทึกการเรียนรู้
}
\lesson{14}
{
\underline{บทที่ 10 การจัดการไฟล์ (Files)}
\begin{itemize}
\item การทำงานกับ Directories
\item การเปิดไฟล์
\item การอ่านไฟล์
\item การจัดการข้อผิดพลาด
\item ฐานข้อมูลแบบ Key-Value
\item การเรียกใช้โปรแกรมอื่น
\end{itemize}
}
{
\item ทดสอบทบทวนความรู้
\item  บรรยายและยกตัวอย่างการเขียนโปรแกรม
\item  ถามตอบในชั้นเรียน
\item  ปฏิบัติการเขียนโปรแกรม
\item  บันทึกการเรียนรู้
}
\lesson{15}
{
\underline{บทที่ 11 Object-Oriented Programming}
\begin{itemize}
\item คลาสและออบเจ็กต์
\item การสร้างคลาส
\item การสร้างออบเจ็กต์
\item ฟังก์ชัน \texttt{\_\_init\_\_()}
\item การสร้างเมธอดของออบเจ็กต์
\item การแก้ไขค่าแอตทริบิวต์ของออบเจ็กต์
\item การลบแอตทริบิวต์ของออบเจ็กต์
\item การลบออบเจ็กต์
\item การสืบทอดคลาส
\end{itemize}

\underline{ทบทวนเนื้อหาก่อนสอบปลายภาค}
}
{
\item ทดสอบทบทวนความรู้
\item  บรรยายและยกตัวอย่างการเขียนโปรแกรม
\item  ถามตอบในชั้นเรียน
\item  ปฏิบัติการเขียนโปรแกรม
\item  บันทึกการเรียนรู้
}


\section*{การจัดการประสบการณ์การเรียนรู้}

\begin{itemize}
\item บรรยายและถ่ายทอดประสบการณ์แก่ผู้เรียน
\item ถามตอบในชั้นเรียน
\item ฝึกปฏิบัติการเขียนโปรแกรม
\item  ฝึกนำเสนอผลงานการเขียนโปรแกรมที่พัฒนาด้วยตนเอง
\item  บันทึกสะท้อนสิ่งที่ได้เรียนรู้
\item  ทดสอบย่อยเพื่อทบทวนความรู้ความเข้าใจในแต่ละบท
\item  สอบปฏิบัติเขียนโปรแกรมเพื่อการบูรณาการความรู้ที่ได้รับทั้งกลางภาคและปลายภาค
\item  ใช้เอกสารประกอบการสอนเพื่อใช้การทบทวนความรู้ที่ได้รับและฝึกทำแบบฝึกหัดท้ายบท
\end{itemize}


\section*{สื่อการเรียนรู้}

\begin{itemize}
\item เอกสารประกอบการสอนรายวิชา 477-201 การเขียนโปรแกรมคอมพิวเตอร์
\item คลิปวิดีโอออนไลน์สื่อการสอนรายวิชา 477-201 การเขียนโปรแกรมคอมพิวเตอร์
    \begin{itemize}
    \item \url{https://www.youtube.com/classstartacademy}
    \end{itemize}
\item ชั้นเรียนออนไลน์รายวิชา 477-201 การเขียนโปรแกรมคอมพิวเตอร์
    \begin{itemize}
    \item \url{https://classstart.org}
    \end{itemize}
\item เว็บไซต์
    \begin{itemize}
    \item \url{https://www.python.org/}
    \item \url{https://code.org/}
    \item \url{https://www.tutorialspoint.com/python3/index.htm}
    \item \url{https://www.w3schools.com/python/}
    \end{itemize}
\end{itemize}

\section*{การประเมินผล}

\begin{itemize}
\item ร้อยละ 30 สอบกลางภาคแบบปฏิบัติการเขียนโปรแกรม
\item ร้อยละ 30 สอบปลายภาคแบบปฏิบัติการเขียนโปรแกรม
\item ร้อยละ 20 แบบฝึกหัดย่อยแบบปฏิบัติการเขียนโปรแกรม
\item ร้อยละ 20 แบบทดสอบย่อยแบบ MCQs
\end{itemize}

\section*{สัปดาห์ที่ทำการประเมินผล}

\begin{tcolorbox}[breakable,enhanced,fonttitle=\bfseries]
\begin{itemize}
\item สัปดาห์ที่ 3
\item วัตถุประสงค์การประเมิน
	\\ เมื่อฟังการบรรยาย ถามตอบในชั้นเรียน และฝึกปฏิบัติการเขียนโปรแกรม Python เกี่ยวกับเนื้อหาบทที่1-2 ความรู้เบื้องต้นเกี่ยวกับ Python และ ส่วนประกอบต่างๆ ของภาษา Python แล้วผู้เรียนสามารถ 
	\begin{itemize}
	\item อธิบายหลักการหน้าที่ของโปรแกรมคอมพิวเตอร์ได้
	\item อธิบายการทำงานของ Python ได้
	\item วิเคราะห์อัลกอริทึมและเขียนผังงานได้
	\item ติดตั้งโปรแกรมจัดการ Python Runtime และ IDE ได้
	\item ใช้ตัวแปรและเครื่องหมายคำนวณทางคณิตศาสตร์ในการเขียนโปรแกรมได้
	\item ใช้คำสั่งพื้นฐาน print และ input ในการเขียนโปรแกรมได้
	\item เขียนโปรแกรมภาษา Python ที่มีโครงสร้างตามลำดับได้
	\end{itemize}
\item วิธีการประเมิน
	\begin{itemize}
	\item แบบฝึกหัดย่อยแบบปฏิบัติการเขียนโปรแกรม
	\item แบบทดสอบย่อยแบบ MCQs
	\end{itemize}
\end{itemize}
\end{tcolorbox}
\vspace{1.5cm}

\begin{tcolorbox}[breakable,enhanced,fonttitle=\bfseries]
\begin{itemize}
\item สัปดาห์ที่ 5
\item วัตถุประสงค์การประเมิน
	\\ เมื่อฟังการบรรยาย ถามตอบในชั้นเรียน และฝึกปฏิบัติการเขียนโปรแกรมภาษา Python เกี่ยวกับเนื้อหาบทที่ 3 ประโยคเงื่อนไขในภาษา Python แล้วผู้เรียนสามารถ
	\begin{itemize}
	\item อธิบายหลักการของ Boolean Expression ได้
	\item เขียนโปรแกรม Python ที่มีโครงสร้างทางเลือกโดยมีเงื่อนไขได้
	\end{itemize}
\item วิธีการประเมิน
	\begin{itemize}
	\item แบบฝึกหัดย่อยแบบปฏิบัติการเขียนโปรแกรม
	\item แบบทดสอบย่อยแบบ MCQs
	\end{itemize}
\end{itemize}
\end{tcolorbox}
\vspace{1.5cm}

\begin{tcolorbox}[breakable,enhanced,fonttitle=\bfseries]
\begin{itemize}
\item สัปดาห์ที่ 7
\item วัตถุประสงค์การประเมิน
	\\ เมื่อฟังการบรรยาย ถามตอบในชั้นเรียน และฝึกปฏิบัติการเขียนโปรแกรม Python เกี่ยวกับเนื้อหาบทที่ 4 การเขียนฟังก์ชันในภาษา Python แล้วผู้เรียนสามารถ
	\begin{itemize}
	\item อธิบายหลักการของฟังก์ชันหรือโปรแกรมย่อย
	\item เข้าใจวิธีการแบ่งโปรแกรมใหญ่เป็นโปรแกรมย่อยและเรียกใช้โปรแกรมย่อยได้
	\item เขียนฟังก์ชันในภาษา Python ได้
	\end{itemize}
\item วิธีการประเมิน
	\begin{itemize}
	\item แบบฝึกหัดย่อยแบบปฏิบัติการเขียนโปรแกรม
	\item แบบทดสอบย่อยแบบ MCQs
	\end{itemize}
\end{itemize}
\end{tcolorbox}
\vspace{1.5cm}

\begin{tcolorbox}[breakable,enhanced,fonttitle=\bfseries]
\begin{itemize}
\item สัปดาห์ที่ 8
\item วัตถุประสงค์การประเมิน
	\\ เมื่อฟังการบรรยาย ถามตอบในชั้นเรียน และฝึกปฏิบัติการเขียนโปรแกรม Python เกี่ยวกับเนื้อหาบทที่ 5 การใช้ประโยคสั่งทำงานซ้ำในภาษา Python แล้วผู้เรียนสามารถ
	\begin{itemize}
	\item เข้าใจกระบวนการทำงานแบบวนซ้ำ
	\item เขียนโปรแกรมภาษา Python โดยใช้โครงสร้างการทำวนซ้ำแบบ \pyinline{for} และ \pyinline{while} ได้
	\item เขียนโปรแกรมภาษา Python โดยใช้ฟังก์ชัน \pyinline{range()} ได้
	\item เขียนโปรแกรมภาษา Python สร้างฟังก์ชันที่เรียกตัวเองได้
	\end{itemize}
\item วิธีการประเมิน
	\begin{itemize}
	\item แบบฝึกหัดย่อยแบบปฏิบัติการเขียนโปรแกรม
	\item แบบทดสอบย่อยแบบ MCQs
	\end{itemize}
\end{itemize}
\end{tcolorbox}
\vspace{1.5cm}

\begin{tcolorbox}[breakable,enhanced,fonttitle=\bfseries]
\begin{itemize}
\item สัปดาห์สอบกลางภาคเชิงปฏิบัติการ
\item วัตถุประสงค์การประเมิน
	\begin{itemize}
	\item เพื่อประเมินความรู้ความเข้าใจทั้งหมดของเนื้อหาบทที่ 1-5
	\end{itemize}
\item วิธีการประเมิน
	\begin{itemize}
	\item สอบปฏิบัติการเขียนโปรแกรม
	\end{itemize}
\end{itemize}
\end{tcolorbox}
\vspace{1.5cm}

\begin{tcolorbox}[breakable,enhanced,fonttitle=\bfseries]
\begin{itemize}
\item สัปดาห์ที่ 10
\item วัตถุประสงค์การประเมิน
	\\ เมื่อฟังการบรรยาย ถามตอบในชั้นเรียน และฝึกปฏิบัติการเขียนโปรแกรม Python เกี่ยวกับเนื้อหาบทที่ 6 ชนิดข้อมูล String แล้วผู้เรียนสามารถ
	\begin{itemize}
	\item อธิบายหลักการใช้และจัดการกับ String และตัวชี้
	\item อธิบายโครงสร้างข้อมูลที่เปลี่ยนแปลงไม่ได้ได้
	\item เขียนโปรแกรมภาษา Python เพื่อจัดการกับ String ได้
	\item อธิบายและใช้เมธอดของ String ได้
	\end{itemize}
\item วิธีการประเมิน
	\begin{itemize}
	\item แบบฝึกหัดย่อยแบบปฏิบัติการเขียนโปรแกรม
	\item แบบทดสอบย่อยแบบ MCQs
	\end{itemize}
\end{itemize}
\end{tcolorbox}
\vspace{1.5cm}

\begin{tcolorbox}[breakable,enhanced,fonttitle=\bfseries]
\begin{itemize}
\item สัปดาห์ที่ 11
\item วัตถุประสงค์การประเมิน
	\\ เมื่อฟังการบรรยาย ถามตอบในชั้นเรียน และฝึกปฏิบัติการเขียนโปรแกรม Python เกี่ยวกับเนื้อหาบทที่ 7 ชนิดข้อมูล List แล้วผู้เรียนสามารถ
	\begin{itemize}
	\item อธิบายหลักการใช้และจัดการกับ List และตัวชี้
	\item อธิบายโครงสร้างข้อมูลที่เปลี่ยนแปลงได้ได้
	\item เขียนโปรแกรมภาษา Python เพื่อจัดการกับ List ได้
	\item อธิบายและใช้เมธอดของ List ได้
	\end{itemize}
\item วิธีการประเมิน
	\begin{itemize}
	\item แบบฝึกหัดย่อยแบบปฏิบัติการเขียนโปรแกรม
	\item แบบทดสอบย่อยแบบ MCQs
	\end{itemize}
\end{itemize}
\end{tcolorbox}
\vspace{1.5cm}

\begin{tcolorbox}[breakable,enhanced,fonttitle=\bfseries]
\begin{itemize}
\item สัปดาห์ที่ 12
\item วัตถุประสงค์การประเมิน
	\\ เมื่อฟังการบรรยาย ถามตอบในชั้นเรียน และฝึกปฏิบัติการเขียนโปรแกรม Python เกี่ยวกับเนื้อหาบทที่ 8 ชนิดข้อมูล Dictionary แล้วผู้เรียนสามารถ
	\begin{itemize}
	\item อธิบายหลักการใช้และจัดการกับ Dictionary และตัวชี้
	\item เขียนโปรแกรมภาษา Python เพื่อจัดการกับ Dictionary ได้
	\item อธิบายและใช้เมธอดของ Dictionary ได้
	\end{itemize}
\item วิธีการประเมิน
	\begin{itemize}
	\item แบบฝึกหัดย่อยแบบปฏิบัติการเขียนโปรแกรม
	\item แบบทดสอบย่อยแบบ MCQs
	\end{itemize}
\end{itemize}
\end{tcolorbox}
\vspace{1.5cm}

\begin{tcolorbox}[breakable,enhanced,fonttitle=\bfseries]
\begin{itemize}
\item สัปดาห์ที่ 13
\item วัตถุประสงค์การประเมิน
	\\ เมื่อฟังการบรรยาย ถามตอบในชั้นเรียน และฝึกปฏิบัติการเขียนโปรแกรม Python เกี่ยวกับเนื้อหาบทที่ 9 ชนิดข้อมูล Tuple แล้วผู้เรียนสามารถ
	\begin{itemize}
	\item อธิบายหลักการใช้และจัดการกับ Tuple และตัวชี้
	\item เขียนโปรแกรมภาษา Python เพื่อจัดการกับ Tuple ได้
	\item อธิบายและใช้เมธอดของ Tuple ได้
	\end{itemize}
\item วิธีการประเมิน
	\begin{itemize}
	\item แบบฝึกหัดย่อยแบบปฏิบัติการเขียนโปรแกรม
	\item แบบทดสอบย่อยแบบ MCQs
	\end{itemize}
\end{itemize}
\end{tcolorbox}
\vspace{1.5cm}

\begin{tcolorbox}[breakable,enhanced,fonttitle=\bfseries]
\begin{itemize}
\item สัปดาห์ที่ 14
\item วัตถุประสงค์การประเมิน
	\\ เมื่อฟังการบรรยาย ถามตอบในชั้นเรียน และฝึกปฏิบัติการเขียนโปรแกรม Python เกี่ยวกับเนื้อหาบทที่ 10 การจัดการ Files แล้วผู้เรียนสามารถ
	\begin{itemize}
	\item อธิบายหลักการใช้และจัดการกับ Files ใน Python ได้
	\item เขียนโปรแกรม Python เพื่อจัดการกับ Files ได้
	\item เขียนโปรแกรม Python เพื่อจัดการกับฐานข้อมูลแบบ Key-Value ได้
	\end{itemize}
\item วิธีการประเมิน
	\begin{itemize}
	\item แบบฝึกหัดย่อยแบบปฏิบัติการเขียนโปรแกรม
	\item แบบทดสอบย่อยแบบ MCQs
	\end{itemize}
\end{itemize}
\end{tcolorbox}
\vspace{1.5cm}

\begin{tcolorbox}[breakable,enhanced,fonttitle=\bfseries]
\begin{itemize}
\item สัปดาห์ที่ 15
\item วัตถุประสงค์การประเมิน
	\\ เมื่อฟังการบรรยาย ถามตอบในชั้นเรียน และฝึกปฏิบัติการเขียนโปรแกรม Python เกี่ยวกับเนื้อหาบทที่ 11 Object-oriented programming (OOP) แล้วผู้เรียนสามารถ
	\begin{itemize}
	\item อธิบายหลักการพื้นฐานเขียนโปรแกรม Python แบบ OOP ได้
	\item อธิบายหลักการทำงานของคลาสและออบเจ็กต์
	\item อธิบายหลักการทำงานแบบ Inheritance และ Polymorphism ได้
	\item เขียนโปรแกรม Python แบบ OOP พื้นฐานได้
	\end{itemize}
\item วิธีการประเมิน
	\begin{itemize}
	\item แบบฝึกหัดย่อยแบบปฏิบัติการเขียนโปรแกรม
	\item แบบทดสอบย่อยแบบ MCQs
	\end{itemize}
\end{itemize}
\end{tcolorbox}
\vspace{1.5cm}

\begin{tcolorbox}[breakable,enhanced,fonttitle=\bfseries]
\begin{itemize}
\item สัปดาห์ปลายกลางภาคเชิงปฏิบัติการ
\item วัตถุประสงค์การประเมิน
	\begin{itemize}
	\item เพื่อประเมินความรู้ความเข้าใจทั้งหมดของเนื้อหาบทที่ 6-11
	\end{itemize}
\item วิธีการประเมิน
	\begin{itemize}
	\item สอบปฏิบัติการเขียนโปรแกรม
	\end{itemize}
\end{itemize}
\end{tcolorbox}
\vspace{1.5cm}

\section*{เอกสารอ้างอิงที่ใช้ในการสอน}

% \renewcommand{\refname}{}
% \patchcmd{\thebibliography}{\section*{\refname}}{}{}{}
\makeatletter
\renewcommand{\chapter}{\@gobbletwo}
\bibliography{book}
\makeatother

\section*{นักศึกษา}

นักศึกษาที่เรียนวิชานี้เป็นนักศึกษาคณะวิทยาการจัดการ ภาควิชาบริหารธุรกิจ สาขาวิชาระบบสารสนเทศ ชั้นปีที่ 2 จำนวน 42 คน

\section*{ผลการสอน}

\begin{table}[H]
\centering
\begin{tabu}{c l r r}
 \hline
 ระดับคะแนน & ช่วงคะแนน & จำนวนผู้เรียน & ร้อยละ \\ [0.5ex] 
 \hline
A & 80 - 100 & 13 & 30.95 \\
B+  & 75 - 79.99  & 6 & 14.29 \\
B  & 70 - 74.99  & 4 & 9.52 \\
C+ & 65 - 69.99 & 7 & 16.67 \\
C  & 60 - 64.99 & 2 & 4.76 \\
D+  & 55 - 59.99 & 2 & 4.76 \\
D  & 50 - 54.99 & 6 & 14.29 \\
E  & 0 - 49.99 & 2 & 4.76 \\
\hline
\end{tabu}
\end{table}


\vspace{10mm}
\noindent
\hangindent=9cm
\hangafter=0
ลงชื่อ\\
\vspace{10mm}\\
\textbf{ดร.จันทวรรณ ปิยะวัฒน์}\\
ผู้สอน



