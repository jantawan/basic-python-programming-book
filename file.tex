\chapter{การจัดการไฟล์ (Files)}
\section{ความหมายของไฟล์}

ไฟล์คือพื้นที่เก็บข้อมูลบนคอมพิวเตอร์ ไฟล์มีหลายประเภทตามการใช้งาน เช่น ไฟล์ Microsoft Word ไฟล์เพลง และไฟล์ VDO เป็นต้น โดยทั่วไปไฟล์มี 2 ประเภท คือ Text files และ Binary files โดย Text files เป็นไฟล์ที่เก็บชุดข้อความซึ่งเปิดอ่านได้ ส่วน Binary files จะอยู่ในรูปแบบของ Binary form เพื่อให้คอมพิวเตอร์ทำงาน

การเขียนโปรแกรมเพื่อใช้งานไฟล์ เช่น อ่านไฟล์ เขียนไฟล์ สร้างไฟล์ และแก้ไขไฟล์ โดยในภาษา Python จะต้องเรียกใช้โมดูล os คือ ระบบปฏิบัติการ (Operating Systems) os module มีความสามารถหลายอย่าง ศึกษาเพิ่มเติมที่ \url{https://docs.python.org/3.5/library/os.html}

\section{การทำงานกับ Directories}

การเรียกใช้โมดูล os ต้องใช้คำสั่ง import os ก่อนแล้วจะสามารถใช้คำนั่งในโมดูลได้ เช่น os.getcwd() ใช้แสดง Directory ที่กำลังทำงานอยู่ ส่วนคำนั่ง os.chdir() คือการเปลี่ยนตำแหน่งการทำงานของ Directory

\begin{codelist}{ฟังก์ชัน os.getcwd() และ os.chdir()}{}
>>> import os
>>> os.getcwd()
|\rq{}c:\textbackslash{}Users\textbackslash{}janta\textbackslash{}AppData\textbackslash{}Local\textbackslash{}Programs\textbackslash{}Python\textbackslash{}Python37-32\rq{}|
>>> os.chdir('c:\Users\janta\Desktop')
\end{codelist}

คำสั่งอื่นๆ เช่น 

\begin{itemize}
\item 	\pyinline{os.path.abspath('file.txt')}
\item 	\pyinline{os.path.exists('file.txt')} 
\item 	\pyinline{os.path.isdir('file.txt')} 
\item 	\pyinline{os.path.isdir(os.getcwd())} 
\item 	\pyinline{os.listdir(os.getcwd())}
\end{itemize}

\begin{codelist}{ตัวอย่างผลของคำสั่งในโมดูล os}{}
>>> os.path.isdir(os.getcwd())
True
>>> os.listdir(os.getcwd())
['DLLs', 'Doc', 'include', 'Lib', 'libs', 'LICENSE.txt', 'NEWS.txt', 'python.exe', 'python3.dll', 'python38.dll', 'pythonw.exe', 'Scripts', 'tcl', 'Tools', 'vcruntime140.dll']
>>>
\end{codelist}


\section{การเปิดไฟล์}

ฟังก์ชัน\pyinline{open()} มีไว้เพื่อเปิดไฟล์ ก่อนที่จะเริ่มทำงานกับไฟล์ทุกครั้งจะต้องทำการเปิดไฟล์ก่อน โดยมีรูปแบบการเขียนคือ \pyinline{fout = open(filename, flag)} ซึ่ง Filename คือ ชื่อไฟล์ที่ต้องการเปิด ส่วน Flag คือ รูปแบบในการเปิดไฟล์ ซึ่งมีหลายแบบ เช่น r ใช้อ่านข้อมูลอย่างเดียว w ใข้เขียนข้อมูลลงในไฟล์ใหม่ที่ไม่ได้สร้างไว้ก่อนหน้า a ใช้เขียนต่อท้ายไฟล์เดิม เป็นต้น และทุกครั้งหลังใช้งานไฟล์เสร็จแล้ว จะต้องทำการปิดไฟล์ด้วยคำสั่ง \pyinline{close()}

\begin{codelist}{การสร้างไฟล์}{}
import os
os.chdir('c:\textbackslash\Users\janta\Desktop')
fout = open('file.txt', 'w')
fout.write('The first line. \n')
fout.write('The second line. \n')
fout.write('The thrid line. \n')
fout.close()
\end{codelist}


\section{การอ่านไฟล์}

คำสั่ง \pyinline{read()} จะทำการอ่านข้อมูลทั้งหมดในไฟล์เพียงครั้งเดียว และข้อมูลที่อ่านได้จะเป็น String ส่วนคำสั่ง \pyinline{readline()} จะทำการอ่านข้อมูลทีละบรรทัดแล้วเก็บไว้เป็น String ทีละบรรทัดใน List

\section{การจัดการข้อผิดพลาด (Error)}

การเขียนโปรแกรมเพื่อให้สามารถจัดการเหตุการณ์ที่คาดไม่ถึงได้ อาทิ เปิดไฟล์ไม่ได้เพราะ Hard disk มีปัญหา หรือ Network มีปัญหา สามารถเขียนโปรแกรมให้แสดง Error message ออกมาได้ โดยใช้ Try and Except statements สำหรับ \pyinline{FileNotFoundError} เช่น ในการเปิดไฟล์ที่ยังไม่ได้สร้างไฟล์ไว้ก่อนหน้า แล้วต้องการให้ Error message แสดงออกมาว่า  File not found!

Source code:
\begin{codelist}{Source code ตัวอย่าง FileNotFoundError}{}
try:
    fin = open('abcdef.txt')
except FileNotFoundError:
    print('File not found.')
\end{codelist}

Result:
\begin{codelist}{Result ตัวอย่าง FileNotFoundError}{}
>>>
File not found.
>>>
\end{codelist}

นอกจากนี้ยังมี finally statement จะทำงานท้ายสุดไม่ว่าจะประสบความสำเร็จหรือไม่ก็ตาม อ่านเพิ่มเติมเรื่อง eexception handling ได้ที่ \url{https://docs.python.org/3/library/exceptions.html\#bltin-exceptions}

Source code:
\begin{codelist}{Source code ตัวอย่าง Finally statement}{}
try:
    fin = open('file.txt')
except:
    print('File not found.')
finally:
    print('Finally...everything is ok.')
\end{codelist}

Result:
\begin{codelist}{Result ตัวอย่าง Finally statement}{}
>>>
Finally...everything is ok.
>>>
\end{codelist}

Source code:
\begin{codelist}{Source code ตัวอย่าง Raise exception}{}
def print_james(name):
    if name == 'James':
        print('Hello James.')
    else:
        raise Exception('Name is not James.')

print_james('Jake')
\end{codelist}

Result:
\begin{codelist}{Result ตัวอย่าง Raise exception}{}
Traceback (most recent call last):
  File "C:\Users\janta\OneDrive\Documents\Python 1_63\01_Overview\code01.py", line 30, in <module>
    print_james('Jake')
  File "C:\Users\janta\OneDrive\Documents\Python 1_63\01_Overview\code01.py", line 28, in print_james
    raise Exception('Name is not James.')
Exception: Name is not James.
\end{codelist}



\section{ฐานข้อมูลแบบ Key-Value}

Databases คือ ไฟล์แบบ Binary ที่เก็บข้อมูลในเครื่องคอมพิวเตอร์ที่เก็บในลักษณะ Key และ Value เหมือน Dictionary โดยจะต้องมีการ import โมดูล dbm ก่อนซึ่งเป็นการจัดการเกี่ยวกับ Database และผลลัพธ์ที่ได้จากการอ่านไฟล์ Database จะมีการแสดงตัวอักษร b ไว้ด้านหน้าเพื่อแสดงความเป็น Binary

Source code:
\begin{codelist}{Source code ตัวอย่าง Key/Value Databases}{}
import dbm
db = dbm.open('stocks.db', 'c')
db['ptt']='360'
db['scb']='160'
print(db['ptt'])
print(db['scb'])
db.close()
\end{codelist}

Result:
\begin{codelist}{Result ตัวอย่าง Key/Value Databases}{}
>>>
b'360'
b'160'
>>>
\end{codelist}


การ Pickling เซฟข้อมูลเป็นอย่างอื่นนอกจาก text ดังนั้นให้แปลงอย่างอื่นให้เป็น text แล้วแปลงเป็น object เพื่อดึงกลับขึ้นมา

เริ่มต้นด้วยการ import pickle เช่น ให้ list ชื่อว่า t1 แล้วทำการ dumps(t1) ให้กลายเป็น string ตอนนี้ก็จะสามารถเอา string ไปเก็บในไฟล์ข้อมูลหรือเก็บใน key-value database ก็ได้ และเมื่อต้องการเรียกนำกลับมาใช้ในสภาพเดิมให้ใช้คำสั่ง loads(s) 

\begin{codelist}{การ Pickling}{}
>>> import pickle
>>> t = [1,2,3,4,5,6,7]
>>> t
[1,2,3,4,5,6,7]
>>> t1 = t
>>> t1
[1,2,3,4,5,6,7]
>>> s = pickle.dumps(t1)
>>> s
 b'\x80\x04\x95\x13\x00\x00\x00\x00\x00\x00\x00]\x94(K\x01K\x02K\x03K\x04K\x05K\x06K\x07e.'
>>> t2 = pickle.loads(s)
>>> t2
[1, 2, 3, 4, 5, 6, 7]
>>> id(t1)
66507840
>>> id(t2)
63748456
\end{codelist}



\section{การให้ Python เรียกใช้โปรแกรมอื่น}

การให้ Python เรียกใช้โปรแกรมอื่นได้ (Piping) โดยใช้โมดูล os และใช้ฟังก์ชัน popen()  อาทิ ตั้งตัวแปรชื่อ cmd เรียกใช้คำสั่ง dir ของ Windows OS ด้วยคำสั่ง popen แล้วนำผลจากคำสั่งนั้นมาเก็บเป็น String ไว้ในตัวแปรชื่อ result 

Source code:
\begin{codelist}{Source code ตัวอย่างการให้ Python เรียกใช้โปรแกรมอื่นได้}{}
import os
os.chdir('c:\\users\\janta\\Desktop')
cmd = 'dir'
fp = os.popen(cmd)
result = fp.read()
fp.close()
print(result)

\end{codelist}

Result:
\begin{codelist}{Result ตัวอย่างการให้ Python เรียกใช้โปรแกรมอื่นได้}{}
Volume in drive C has no label.
Volume Serial Number is 2E09-DC35
Directory of c:\users\janta\Desktop
07/05/2020  01:34 PM    <DIR>          .
07/05/2020  01:34 PM    <DIR>          ..
07/03/2020  05:46 PM            52,422 560gra01.png
07/05/2020  01:34 PM            29,887 5f83b4e260d668989b715e8bf0e3e288.jpg
07/02/2020  07:04 PM           299,334 latexsheet-a4.pdf
07/03/2020  06:22 PM        14,832,375 Manual_MS-Teams_Full.pdf
07/03/2020  06:11 PM             2,394 Microsoft Teams.lnk
               5 File(s)     15,216,412 bytes
               2 Dir(s)  90,572,718,080 bytes free
\end{codelist}



\section{แบบฝึกหัด}
\begin{enumerate} 
\item 	เขียนฟังก์ชันอ่าน text file ทั้งไฟล์
\item 	เขียนฟังก์ชันอ่านไฟล์ n บรรทัดแรกใน text file
\item 	เขียนฟังก์ชันอ่านไฟล์ทีละบรรทัดแล้วเก็บไว้ใน list
\item 	สร้าง text file ขึ้นมา แล้วเพิ่มบรรทัดใหม่หนึ่งบรรทัดว่า Welcome to my class. แล้ว print เนื้อหาออกมาทั้งหมด
\end{enumerate}



