\chapter{คำนำ}

เอกสารประกอบการสอนเล่มนี้จัดทำขึ้นสำหรับการสอนรายวิชา 477-201 การเขียนโปรแกรมคอมพิวเตอร์ (Computer Programming) ในภาคการศึกษาที่ 1 ปีการศึกษา 2563 ซึ่งเป็นรายวิชาบังคับของนักศึกษาหลักสูตรระบบสารสนเทศทางธุรกิจ ชั้นปีที่ 2 ภาควิชาบริหารธุรกิจ คณะวิทยาการจัดการ มหาวิทยาลัยสงขลานครินทร์ วิทยาเขตหาดใหญ่ จำนวน 3 หน่วยกิต 3(2-2-5) เป็นการสอนทฤษฎี 2 ชั่วโมงต่อสัปดาห์ ปฏิบัติ 2 ชั่วโมงต่อสัปดาห์ และนักศึกษาควรศึกษาค้นคว้าด้วยตัวเอง 5 ชั่วโมงต่อสัปดาห์

วิชา 477-201 การเขียนโปรแกรมคอมพิวเตอร์ มีจุดมุ่งหมายให้นักศึกษาได้มีพื้นฐานความรู้ความเข้าใจในหลักการเขียนโปรแกรมคอมพิวเตอร์เบื้องต้นด้วยภาษา Python ส่วนประกอบต่างๆของโปรแกรมคอมพิวเตอร์และภาษา Python สามารถเขียนโปรแกรมคอมพิวเตอร์พื้นฐานด้วยภาษา Python ได้ตามการวิเคราะห์และออกแบบขั้นตอนการทำงานของโปรแกรมอย่างมีระบบ สามารถเขียนโปรแกรมแบบมีเงื่อนไขเพื่อการตัดสินใจ เขียนคำสั่งเพื่อให้โปรแกรมทำงานวนซ้ำได้ และเข้าใจการใช้งานโมดูลส่วนเสริมต่างๆ ของโปรแกรมภาษา Python เพื่อนำความรู้เหล่านี้ไปใช้ในการเขียนโปรแกรมระดับในระดับที่ยากขึ้นซึ่งได้แก่ การเขียนโปรแกรมแบบฟังก์ชันและการเขียนโปรแกรมเชิงวัตถุได้

หนังสือเล่มนี้ได้จัดแบ่งเนื้อหาออกเป็น 11 บท ในแต่ละบทจะมีแบบฝึกหัดท้ายบทเพื่อให้ผู้เรียนได้ลองวิเคราะห์และออกแบบแนวทางแก้ไขปัญหาและพัฒนาออกมาเป็นโปรแกรมด้วยภาษา Python ที่ได้เรียนรู้ไปแล้วได้ ทั้งนี้ผู้จัดทำหวังเป็นอย่างยิ่งว่าเอกสารประกอบการสอนฉบับนี้จะให้ความรู้และเป็นประโยชน์แก่ผู้เรียนและผู้อ่านทุกๆ ท่าน เพื่อสร้างความรู้ความเข้าใจในการฝึกเขียนโปรแกรมคอมพิวเตอร์เบื้องต้นให้ดียิ่งขึ้น หากมีข้อเสนอแนะประการใด ผู้จัดทำขอรับไว้ด้วยความขอบพระคุณยิ่ง

\vspace{10mm}
\noindent
\textbf{ดร.จันทวรรณ ปิยะวัฒน์}\\
สาขาวิชาระบบสารสนเทศทางธุรกิจ\\
ภาควิชาบริหารธุรกิจ คณะวิทยาการจัดการ\\
มหาวิทยาลัยสงขลานครินทร์