\chapter{ส่วนประกอบต่าง ๆ ของภาษา Python}

\section{ตัวแปร (Variables)}

ตัวแปร (Variables) คือชื่อที่กำหนดขึ้นสำหรับใช้เก็บค่าในหน่วยความจำของเครื่องคอมพิวเตอร์

\section{การตั้งชื่อตัวแปร}

การตั้งชื่อตัวแปรมีเงื่อนไขดังนี้

\begin{enumerate}[noitemsep]
\item ให้ขึ้นต้นด้วยอักษรตัวภาษาอังกฤษตัวใหญ่หรือตัวเล็กตั้งแต่ Aa ถึง Zz เท่านั้น 
\item ประกอบด้วยตัวอักษรหรือตัวเลข 0 ถึงเลข 9 หรือตัวขีดล่าง Underscore (\_) แต่ห้ามมีช่องว่าง
\item ตัวเลข 0-9 จะนำหน้าชื่อตัวแปรไม่ได้
\item ตัวพิมพ์เล็กและตัวพิมพ์ใหญ่เป็นตัวแปรคนละตัวกัน (Case-Sensitive) เช่น Name ไม่ใช่ตัวแปรเดียวกันกับ name และจะใช้ใส่เครื่องหมาย = ในการตั้งตัวแปรหรือให้ค่าแก่ตัวแปร นอกจากนี้การตั้งชื่อตัวแปรควรตั้งอย่างสมเหตุสมผล อีกทั้ง ภาษา Python จะมีคำที่ถูกสงวนไว้ในการเขียนโปรแกรม หรือ Keywords ซึ่งห้ามนำมาใช้ในการตั้งชื่อตัวแปร ชื่อฟังก์ชัน หรือ ชื่อคลาส มีดังต่อไปนี้ \cite{lutz2014}
\end{enumerate}

\begin{pycode}
>>> a
1
>>> id(a)
1538021648
\end{pycode}