\begin{appendices}

\chapter{ประวัติผู้เขียน}

\section*{ชื่อ-นามสกุล}

ดร.จันทวรรณ ปิยะวัฒน์ (Jantawan Piyawat, Ph.D.)

\section*{สังกัด}

คณะวิทยาการจัดการ มหาวิทยาลัยสงขลานครินทร์

\section*{การศึกษา}

\begin{itemize}
	\item Ph.D. (Information Systems) University of Maryland, Baltimore County, Maryland, USA พ.ศ. 2545
	\item M.S. (Information Systems) University of Maryland, Baltimore County, Maryland, USA พ.ศ. 2543
	\item บธ.ม. (Business Communication – International program) มหาวิทยาลัยหอการค้าไทย พ.ศ. 2539
	\item บธ.บ. (คอมพิวเตอร์ธุรกิจ) มหาวิทยาลัยสงขลานครินทร์ พ.ศ. 2537
\end{itemize}

\section*{ผลงานบริการวิชาการที่พัฒนาขึ้นและมีการใช้งานอย่างต่อเนื่อง}

\begin{itemize}
	\item แพลตฟอร์มออนไลน์เพื่อการจัดการความรู้ \textbf{GotoKnow} (\url{https://GotoKnow.org})
	\item แพลตฟอร์มการเรียนการสอนออนไลน์ \textbf{ClassStart} (\url{https://ClassStart.org})
\end{itemize}

\section*{งานบริหาร}

\begin{itemize}
	\item กรรมการนโยบายหลักสูตรบริหารธุรกิจมหาบัณฑิต พ.ศ. 2562-ปัจจุบัน 
	\item รองผู้อำนวยการศูนย์พัฒนานวัตกรรมเพื่อการจัดการความรู้และการเรียนรู้ คณะวิทยาการจัดการ  พ.ศ. 2554-2559 
\end{itemize}

\section*{การเป็นวิทยากรอบรมและบรรยายพิเศษ}

\begin{itemize}
	\item การใช้เทคโนโลยีการจัดการเรียนการสอน (whole school management system) แบบออนไลน์ ชื่อ ClassStart.org ใน 29 - 30 กันยายน 2562  จังหวัดสตูล
	\item การนำเข้าข้อมูลโครงการเพาะพันธุ์ปัญญาผ่านการใช้ระบบห้องเรียนออนไลน์ ClassStart.org 17-18 สิงหาคม  จังหวัดกรุงเทพมหานคร
	\item การเรียนรู้เครื่องมือเทคโนโลยีสารสนเทศเพื่อการปฏิรูปการศึกษา 5-8 กรกฎาคม  จังหวัดภูเก็ต
	\item การนำเข้าข้อมูลโครงการเพาะพันธุ์ปัญญาผ่านการใช้ระบบห้องเรียนออนไลน์ ClassStart.org 30 มิถุนายน – 1 กรกฎาคม 2652 ง จังหวัดสงขลา
	\item การจัดการเรียนการสอนรายวิชาศึกษาทั่วไปแบบ Flipped Classroom 6-7 มิถุนายน 2562  จังหวัดขอนแก่น
	\item การเรียนรู้เครื่องมือเทคโนโลยีสารสนเทศเพื่อการปฏิรูปการศึกษา 2-3 พฤษภาคม 2562  จังหวัดสงขลา
	\item การจัดการเรียนการสอนแบบ Flipped Classroom 29-30 เมษายน 2562  จังหวัดสงขลา
	\item GotoKnow.org ความรู้คู่ความยั่งยืน 15 มีนาคม 2562  จังหวัดกรุงเทพมหานคร
	\item การเรียนรู้เครื่องมือเทคโนโลยีสารสนเทศเพื่อการปฏิรูปการศึกษา 30 พฤศจิกายน 2561จังหวัดสงขลา
	\item การพัฒนาระบบสารสนเทศเพื่อการจัดการการศึกษาในระดับประเทศ 30 ตุลาคม 2561  จังหวัดกรุงเทพมหานคร
	\item การใช้เครื่องมือสารสนเทศในการออกแบบกิจกรรมการเรียนการสอนและการสร้างรายวิชาออนไลน์ 2 พฤษภาคม 2561 จังหวัดสงขลา
	\item KM-GotoKnow.org สานต่อ คุณธรรม คุณค่า และคุณภาพ สังคมออนไลน์ 15 มีนาคม 2561 จังหวัดกรุงเทพมหานคร
	\item Flipped classroom 19 ตุลาคม 2560  จังหวัดสงขลา
	\item Flipped classroom - ClassStart.org 23-24 มีนาคม 2560  จังหวัดนครปฐม
	\item Online Technology เพื่อการจัดการความรู้ 23 กุมภาพันธ์ 2560  จังหวัดสงขลา
	\item HA\_KM เติมเต็มจากใจชาว GotoKnow.org 16 มีนาคม 2560 จังหวัดกรุงเทพมหานคร
	\item การจัดการเรียนรู้ทักษะในศตวรรษที่ 21 31 สิงหาคม 2559 จังหวัดสงขลา
	\item KM Version 3.0: การพัฒนานักศึกษา 22-23 สิงหาคม 2559  จังหวัดสงขลา
	\item ระบบชั้นเรียนออนไลน์ ClassStart.org แบบห้องเรียนกลับทาง 26 กรกฎาคม 2559  จังหวัดเชียงใหม่
	\item Flipped Classroom 27 มิถุนายน 2559  จังหวัดตรัง
	\item การจัดการเรียนการสอนแบบ Active Learning 27 พฤกษาคม 2559 จังหวัดสงขลา
	\item Knowledge Management: เรียนรู้ทุกขณะจิต 9 มีนาคม 2559  จังหวัดกรุงเทพมหานคร
	\item Flipped Classroom 17 ธันวาคม 2558 จังหวัดสงขลา
	\item การจัดการเรียนการสอนแบบ Active Learning 3 ธันวาคม 2558  จังหวัดชลบุรี
	\item โชว์และแชร์ 1: สร้างพื้นที่การสื่อสารเพื่อแลกเปลี่ยนเรียนรู้ Online และ Offline  16 ตุลาคม 2558 จังหวัดนนทบุรี
	\item Flipped classroom with ClassStart.org 19 สิงหาคม 2558  จังหวัดนครศรีธรรมราช
\end{itemize}

\section*{ทุนวิจัยและพัฒนา}

\begin{itemize}
	\item พ.ศ. 2554-2557 ชุมชนออนไลน์เพื่อการจัดการความรู้ GotoKnow.org สนับสนุนโดย ศูนย์เรียนรู้สุขภาวะ สำนักงานกองทุนสนับสนุนการสร้างเสริมสุขภาพ (สสส.)
	\item พ.ศ. 2551-2554 โครงการระบบออนไลน์เพื่อการจัดการความรู้สุขภาวะ สนับสนุนโดย สำนักงานกองทุนสนับสนุนการสร้างเสริมสุขภาพ (สสส.)
	\item พ.ศ. 2552-2553 โครงการพัฒนาเว็บไซต์กลางของมหาวิทยาลัยสงขลานครินทร์ สนับสนุนโดย มหาวิทยาลัยสงขลานครินทร์
	\item พ.ศ. 2552-2554 โครงการพัฒนาระบบฐานข้อมูลการเมืองการปกครองแห่งสถาบันพระปกเกล้าฯ สนับสนุนโดย สถาบันพระปกเกล้าฯ
	\item พ.ศ. 2550-2552 โครงการพัฒนาเว็บไซต์ฝ่ายวิชาการ สกว. และชุมชนออนไลน์สำหรับนักวิจัย สกว. สนับสนุนโดย สำนักงานกองทุนสนับสนุนการวิจัย (สกว.)
	\item พ.ศ. 2548-2551 ชุมชนออนไลน์เพื่อการจัดการความรู้ GotoKnow.org สนับสนุนโดย สถาบันส่งเสริมการจัดการความรู้เพื่อสังคม (สคส.)
	\item พ.ศ. 2548-2550 KnowledgeVolution: User-Centered Knowledge Management System Implemented on a Community-Based Web Service Framework สนับสนุนโดย สำนักงานกองทุนสนับสนุนการวิจัย (สกว.)
\end{itemize}

\section*{งานวิจัยที่ตีพิมพ์}

\begin{itemize}
	\item จันทวรรณ ปิยะวัฒน์. (2563). ผู้หญิงก็เขียนโปรแกรมได้: ความแตกต่างระหว่างเพศในด้านพฤติกรรมการเรียนผลสัมฤทธิ์ทางการเรียนและเกรดในรายวิชาการเขียนโปรแกรมคอมพิวเตอร์. การประชุมวิชาการระดับชาติด้านการบริหารจัดการ ครั้งที่ 12 ประจำปี 2563 ณ คณะวิทยาการจัดการ มหาวิทยาลัยสงขลานครินทร์. 6 มิถุนายน 2563.
	\item Jantawan Piyawat. (2018). Sustainable Development of a Knowledge Management Online Community: Lessons Learned from 13 Years of GotoKnow.org. International Conference on Innovations in Interdisciplinary Research (ICIIR). Suratthani Rajabhat University, Suratthani, Thailand. Dec 13-14, 2018. Pages 91-102.
	\item Jantawan Piyawat. (2018). Thailand Education Reform: A Web Content Analysis of Thailand’s Largest Online Community in Education. The 5th National and 3rd International Conference on Education (NICE) 2018. Prince of Songkla University, Phuket, Thailand. July 5-7, 2018. Pages 144-153.
	\item จันทวรรณ ปิยะวัฒน์. (2561). 12 ปีของ GotoKnow.org: การวิเคราะห์ระยะยาวของเว็บไซต์ที่มีข้อมูลและการใช้งานมาก. การประชุมวิชาการระดับชาติด้านการบริหารจัดการ ครั้งที่ 10 ประจำปี 2561 ณ คณะวิทยาการจัดการ มหาวิทยาลัยสงขลานครินทร์. 30 มิถุนายน 2561. หน้า 424-439. 
	\item สุนทรี แซ่ตั่น, จันทวรรณ ปิยะวัฒน์ และ ยุวนุช ทินนะลักษณ์ (2555). การวิเคราะห์ปัจจัยเชิงสำรวจอุปสรรคในการแลกเปลี่ยนเรียนรู้ทางออนไลน์ที่มีผลต่อความตั้งใจในการแลกเปลี่ยนเรียนรู้ของผู้ใช้งานในชุมชนออนไลน์ GotoKnow.org. ประชุมวิชาการระดับชาติด้านการบริหารจัดการ ครั้งที่ 4 (หน้า 587-599).
Noiwan, J. and Norcio, A. F. (2006) “Cultural Differences on Attention and Perceived Usability: Investigating Color Combinations of Animated Graphics", International Journal of Human-Computer Studies, Vol. 64, Issue 2, Pages 103-122 (February 2006).
	\item Noiwan, J., Piyawat, T., and Norcio, A.F. (2005) “Computer Attitude and Computer Self-Efficacy: A Case Study of Thai Undergraduate Students", HCI International 2005, Las Vegas, USA.
	\item Piyawat, T., Noiwan, J., and Norcio, A. F. (2005) “Usability Testing of BlogExpress: Blog Reader Software as a Knowledge Management Tool", HCI International 2005, Las Vegas, USA.
	\item Noiwan, J., and Emurian, H.H. (2001) Effects of target density and graphics presentation style on search time and reports of usability and preference, In M. Khosrowpour (Ed.) Managing Information Technology in a Global Economy, Hershey: Idea Group Publishing, 134-138.
	\item Noiwan, J. and Norcio, A. F. (2001) Designing for Effective Information Presentation: The Effects of Cultural Differences on Speed, Accuracy, and Perceptions on Usability and Aesthetics, Proceedings of the HCI International 2001 Conference, Volume 1 – Usability Evaluation and Interface Design: Cognitive Engineering, Intelligent Agents, and Virtual Reality, 115-119.
	\item Noiwan, J., Piyawat, T., and Norcio, A. F. (2001) The Impact of Culture in Designing Web-Based Systems, Proceedings of the WebNet 2001-World Conference on the WWW and Internet, Orlando, FL, 913-918.
	\item Noiwan, J., Piyawat, T., and Norcio, A. F. (2001) Between-Page Banner Advertising (BePBA) on the Web: A Solution Where Usability and Advertising Meet, Proceedings of the WebNet 2001-World Conference on the WWW and Internet, Orlando, FL, 919-924.
	\item Piyawat, T., Noiwan, J., and Norcio, A. F. (2001) Navigation is Law: The Conceptual Architecture for Layered Web-based Information Systems, Proceedings of the WebNet 2001-World Conference on the WWW and Internet, Orlando, FL, 1011-1012.
	\item Piyawat, T., Noiwan, J., and Norcio, A. F. (2001) Measuring Web Navigational Capability Using Paths, Distances, Loops, and Routes – the WebStruct Tool, Proceedings of the WebNet 2001–World Conference on the WWW and Internet, Orlando, FL, 1013.
	\item Zaphiris, P. (chair), Noiwan, J., Kurniawan, S. H., and Karoulis, A. (2001) Panel on Special Topics of Web Usability. Proceedings of the WebNet 2001-World Conference on the WWW and Internet, Orlando, FL, 1392-1395.
\end{itemize}

\end{appendices}