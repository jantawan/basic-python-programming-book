\chapter{การใช้ประโยคสั่งทำงานวนซ้ำ}
\section{ฟังก์ชัน range()}

ฟังก์ชัน range() คือ ระยะตั้งแต่เริ่มต้นถึงก่อนระยะสิ้นสุด มักจะใช้ในการควบคุมการทำงานของโปรแกรมเป็นจำนวนรอบ มีวิธีการเขียนดังนี้ range(start, end)

\section{คำสั่ง for}

คำสั่ง for statement เป็นการทำงานซ้ำๆ ตามจำนวนครั้งที่ระบุไว้ เช่น การใช้ for statement ร่วมกัน range() โดยมีรูปแบบการเขียนดังนี้


\section{คำสั่ง while}

while Statement เป็นคำสั่งให้โปรแกรมทำงานวนซ้ำในขณะที่เงื่อนไขของการวนซ้ำนั้นยังคงเป็นจริงอยู่ และเมื่อเงื่อนไขเป็นเท็จจะสิ้นสุดการทำงานวนซ้ำนั้นทันที ดังนั้นจึงต้องมีตัวควบคุมในการเพิ่มค่าไปเรื่อยๆ จนเงื่อนไขเป็นเท็จ มีลักษณะการเขียนดังนี้


\section{คำสั่ง break}

คำสั่ง break เพื่อให้หลุดการทำงานในลูป

\section{ฟังก์ชันที่เรียกตัวเอง (Recursion)}

Recursion คือการเรียกใช้ฟังก์ชันซ้อนฟังก์ชันนั้นๆ เอง หรือ ฟังก์ชันเรียกใช้ตัวมันเอง จากฟังก์ชัน countdown() ในตัวอย่าง เมื่อมีการเรียกฟังก์ชัน countdown(5) โปรแกรมจะทำงานลดหลั่นลงไปเรื่อยๆ คือตั้งแต่ 5, 4, 3, 2, 1 ตราบเท่าที่ n ยังมากกว่า 0 แต่เมื่อเงื่อนไขเป็นเท็จแล้ว โปรแกรมจะแสดงคำว่า Go! แทน


\section{แบบฝึกหัด}
\begin{enumerate} 

\item 	สร้างฟังก์ชันหาค่าของ \pyinline{ \[ 3^1 + 3^2 + 3^3 + 3^4 + 3^5 \] } โดยใช้ For loop
\item 	สร้างฟังก์ชันให้ผู้ใช้ป้อนค่า x แล้วนำค่า x มาคำนวณ \pyinline{x^1+x^2+x^3+x^4+x^5}
\item 	สร้างฟังก์ชันให้ผู้ใช้ป้อนค่า n และให้แสดงเลขเริ่มที่ n โดยลดลงทีละหนึ่ง โดยใช้ While loop
\item 	สร้างฟังก์ชันให้ผู้ใช้ป้อนตัวเลข แล้วหาว่ามีเลขใดที่สามารถหารเลขที่ผู้ใช้ป้อนได้ลงตัว เช่น ผู้ใช้ป้อนตัวเลข 4 จะมี 1 2 4 ที่หารเลข 4 ลงตัว
\item 	สร้างฟังก์ชันคำนวณปีเกิด คศ. เป็น 12 ราศีปีนักษัตร
\item 	สร้างฟังก์ชันรับจำนวนเงินมาหนึ่งค่า แล้วแลกเปลี่ยนธนบัตร 100 บาท 50 บาท 20 บาท เหรียญ 10 บาท 5 บาท และ 1 บาท
\end{enumerate}