\chapter{การใช้ประโยคสั่งทำงานวนซ้ำ}
\section{ฟังก์ชัน \texttt{range()}}

ฟังก์ชัน  \pyinline{range()} คือ ระยะตั้งแต่เริ่มต้นถึงก่อนระยะสิ้นสุด มักจะใช้ในการควบคุมการทำงานของโปรแกรมเป็นจำนวนรอบ มีวิธีการเขียนดังนี้  \pyinline{range(start, stop[,step]):} โดย start เป็นค่าเริ่มต้น stop เป็นตำแหน่งค่าระยะสิ้นสุด และ step คือ ระยะห่างการเพิ่มดังตัวอย่างต่อไปนี้

\begin{codelist}{ตัวอย่างการใช้ \texttt{range}}{}
>>> for i in range(5): print(i, end=' ')
0 1 2 3 4 
>>> for i in range(1,5): print(i, end=' ')
1 2 3 4 
>>> for i in range(1,5,2): print(i, end=' ')
1 3 
>>> for i in range(5,1,-2): print(i, end=' ')
5 3 
>>> 
\end{codelist}

\section{คำสั่ง \texttt{for}}

คำสั่ง  \pyinline{for} statement เป็นการทำงานซ้ำๆ ตามจำนวนครั้งที่ระบุไว้อย่างแน่นอน เช่น การใช้  \pyinline{for} statement ร่วมกัน  \pyinline{range()} โดยมีรูปแบบการเขียนดังนี้

\begin{codelist}{รูปแบบการเขียน for statement}{}
for var in sequence:
    # statements
\end{codelist}

ตัวอย่างการใช้ \pyinline{for}  statement เป็นดังนี้

\begin{codelist}{ตัวอย่างการใช้ for statement}{}
for x in range(0,5): print(x, end=' ')
\end{codelist}

ซึ่งจะได้ผลลัพธ์ดังนี้

\begin{codelist}{ผลลัพธ์จากตัวอย่างการใช้ for statement}{}
>>>
0 1 2 3 4
>>>
\end{codelist}

ตัวอย่างการใช้ \pyinline{for}  statement ใน Python shell จะได้ผลลัพธ์ดังนี้

\begin{codelist}{ตัวอย่างการใช้ for statement}{}
>>> name = 'Jan'
>>> for char in name: print(char, end=' ')
J a n 
>>> mylist = ['Jan', 200, 4.5]
>>> for item in mylist: print(item, end = ' ')
Jan 200 4.5 
>>> 
\end{codelist}

ตัวอย่างการใช้ \pyinline{for}  statement ที่มีการผสมของการใช้ \texttt{if} ในการกำหนดเงื่อนไขการตัดสินใจ เป็นดังนี้
\begin{codelist}{ตัวอย่างการใช้ for statement}{}
message = input('Enter a message: ')
vowels = 'AEIOU'
counter = 0
for char in message:
    if char.upper() in vowels:
        counter += 1
print('Vowels: ', counter)
\end{codelist}


ซึ่งจะได้ผลลัพธ์ดังนี้

\begin{codelist}{ผลลัพธ์จากตัวอย่างการใช้ for statement}{}
Enter a message: Jantawan
Vowels:  3
>>> 
\end{codelist}




\section{คำสั่ง \texttt{while}}

 \pyinline{while} Statement เป็นคำสั่งให้โปรแกรมทำงานวนซ้ำในขณะที่เงื่อนไขของการวนซ้ำนั้นยังคงเป็นจริงอยู่ และเมื่อเงื่อนไขเป็นเท็จจะสิ้นสุดการทำงานวนซ้ำนั้นทันที ดังนั้นจึงต้องมีตัวควบคุมในการเพิ่มค่าไปเรื่อยๆ จนเงื่อนไขเป็นเท็จ 
 
 \pyinline{while} Statement มีลักษณะการเขียน 3 รูปแบบ คือ แบบ Pre-Test แบบ Post-Test และแบบ Mid-Test
 
รูปแบบการเขียน  \pyinline{while} statement แบบ Pre-Test เป็นดังนี้คือ

\begin{codelist}{รูปแบบการเขียน while statement}{}
while expression:
    # statements
\end{codelist}

และมีตัวอย่างการใช้ \texttt{while} statement เป็นดังนี้

\begin{codelist}{ตัวอย่างการใช้ while statement}{}
x = 0
while x < 10:
    print(x)
    x = x + 1
\end{codelist}

ซึ่งได้ผลลัพธ์เช่นนี้

\begin{codelist}{ผลลัพธ์จากตัวอย่างการใช้ while statement}{}
0
1
2
3
4
5
6
7
8
9
\end{codelist}

รูปแบบการเขียน  \pyinline{while} statement แบบ Post-Test เป็นดังนี้คือ

\begin{codelist}{รูปแบบการเขียน while statement}{}
while True:
    # statements
    if Boolean_Expression: break
\end{codelist}

และมีตัวอย่างการใช้ \texttt{while} statement เป็นดังนี้

\begin{codelist}{รูปแบบการเขียน while statement}{}
i = 10
while True:
    if i % 8 == 0: print( i, 'can be divided by 8.')
    if i % 4 == 0: print( i, 'can be divided by 4.')
    if i % 2 == 0: print( i, 'can be divided by 2.')
    i -= 2
    if i <= 0: break
\end{codelist}

ซึ่งได้ผลลัพธ์เช่นนี้
\begin{codelist}{ผลลัพธ์จากตัวอย่างการใช้ break}{}
10 หาร 2 ลงตัว
8 หาร 8 ลงตัว
8 หาร 4 ลงตัว
8 หาร 2 ลงตัว
6 หาร 2 ลงตัว
4 หาร 4 ลงตัว
4 หาร 2 ลงตัว
2 หาร 2 ลงตัว
>>>
\end{codelist}

รูปแบบการเขียน  \pyinline{while} statement แบบ Mid-Test เป็นดังนี้คือ

\begin{codelist}{รูปแบบการเขียน while statement}{}
while expression:
    # statements 1
    if Boolean_Expression: break
    # statements 2
    
\end{codelist}

และมีตัวอย่างการใช้ \texttt{while} statement เป็นดังนี้

\begin{codelist}{Mid-test}{}
x = 0
while True:
    print(x)
    if x == 10: break
    x = x +1
\end{codelist}

ซึ่งได้ผลลัพธ์เช่นนี้
\begin{codelist}{ผลลัพธ์จากตัวอย่างการใช้ break}{}
0
1
2
3
4
5
6
7
8
9
10
\end{codelist}





\section{แบบฝึกหัด}
\begin{enumerate} 

\item 	สร้างฟังก์ชันหาค่าของ  $3^1$ + $3^2$ + $3^3$ + $3^4$ + $3^5$  โดยใช้ For loop
\item 	สร้างฟังก์ชันให้ผู้ใช้ป้อนค่า x แล้วนำค่า x มาคำนวณ $x^1$ + $x^2$ + $x^3$ + $x^4$ + $x^5$
\item 	สร้างฟังก์ชันให้ผู้ใช้ป้อนค่า n และให้แสดงเลขเริ่มที่ n โดยลดลงทีละหนึ่ง โดยใช้ while loop
\item 	สร้างฟังก์ชันให้ผู้ใช้ป้อนตัวเลข แล้วหาว่ามีเลขใดที่สามารถหารเลขที่ผู้ใช้ป้อนได้ลงตัว เช่น ผู้ใช้ป้อนตัวเลข 4 จะมี 1 2 4 ที่หารเลข 4 ลงตัว
\item 	สร้างฟังก์ชันคำนวณปีเกิด คศ. เป็น 12 ราศีปีนักษัตร
\item 	สร้างฟังก์ชันรับจำนวนเงินมาหนึ่งค่า แล้วแลกเปลี่ยนธนบัตร 100 บาท 50 บาท 20 บาท เหรียญ 10 บาท 5 บาท และ 1 บาท
\end{enumerate}