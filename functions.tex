\chapter{การเขียนและใช้งานฟังก์ชัน (Functions)}
\section{การเรียกใช้ฟังก์ชัน}

ฟังก์ชัน (Function) คือชุดคำสั่งส่วนหนึ่งของโปรแกรมที่ทำงานเพื่อวัตถุประสงค์บางอย่าง ฟังก์ชันในภาษา Python มีทั้ง Built-in ฟังก์ชันของภาษา Python เอง และฟังก์ชันที่ผู้เขียนโปรแกรมเขียนขึ้นมาเอง การใช้ฟังก์ชันในภาษา Python จะคล้ายกับฟังก์ชันทางคณิตศาสตร์ คือ กำหนดชื่อฟังก์ชันตามด้วยสิ่งที่อยู่ในวงเล็บซึ่งเรียกว่า Arguments ซึ่งอาจจะมีได้มากกว่า 1 และในภาษา Python มีการกำหนดฟังก์ชันมาให้เรียกใช้ได้เลยอยู่บ้างแล้ว เช่น \texttt{type(42)} คือ การแสดงค่าประเภทของเลข 42 หรือ \texttt{id(42)} คือการแสดงตำแหน่งที่อยู่ของเลข 42 ในหน่วยความจำ

\begin{codelist}{ตัวอย่าง Built-in ฟังก์ชันในภาษา Python}{}
>>> type(42)
|<class \rq{}int\rq{}>|
>>> a = 1
>>> id(a)
1538021648
\end{codelist}


\section{การเรียกใช้โมดูล (Modules)}

โมดูล (Modules) คือ ฟังก์ชันที่รวมกันไว้เป็นหมวดหมู่ และสามารถดึงมาใช้ได้ในโปรแกรมได้ด้วยการ \texttt{import} เช่น \texttt{import math} และหากเรียกใช้ฟังก์ชัน \texttt{dir(math)} จะแสดงฟังก์ชันในโมดูล math ออกมา 

\begin{codelist}{การเรียกใช้โมดูล math}{}
>>> import math
>>> type(math)
|<class \rq{}module\rq{}>|
>>> id(math)
56337632

\end{codelist}


การใช้งานโมดูลจะมีการใช้งานแบบ Dot Notation หากเห็นการเขียนโมดูล math ในลักษณะนี้ เช่น  \texttt{math.pi} ตัว  \texttt{pi} เรียกว่าเป็นตัวแปรที่อยู่ในโมดูล  \texttt{math} ซึ่งไม่ใช่ฟังก์ชัน แต่ถ้าเขียน  \texttt{math.pow(2,2)} คือ สองยกกำลังสอง ลักษณะนี้จะเป็นการเรียกใช้ฟังก์ชันที่อยู่ในโมดูล

\begin{codelist}{การใช้งานโมดูลแบบ Dot Notation}{}
>>> math.pi
3.141592653589793
>>> math.pow(2,2)
4.0
\end{codelist}

%\section{ฟังก์ชันซ้อน (Composition)}
การใช้ฟังก์ชันไม่จำเป็นต้องใช้ฟังก์ชันแบบฟังก์ชันเดียว ฟังก์ชันใช้ฟังก์ชันซ้อนกันได้ คือ การรวมฟังก์ชันหลายๆ อันซ้อนกัน เรียกต่อๆ กันไปได้

\begin{codelist}{การเรียกใช้ฟังก์ชันแบบ Composition}{}
>>> math.exp(math.log(3+2))
4.9999999999999
\end{codelist}

\section{การสร้างฟังก์ชันในภาษา Python}

การเขียนฟังก์ชันหรือสร้างฟังก์ชันขึ้นมาเองจะกระทำเพื่อต้องการทำงานเพื่อวัตถุประสงค์บางอย่าง ต้องใช้ \texttt{def} Statement ซึ่งย่อมาจาก Define เมื่อเขียนฟังก์ชันไว้ใน Source Code แล้วทำการ Run ฟังก์ชันนั้นจะถูก Define ไว้ในระบบแล้วถูกเรียกใช้ขึ้นมาได้เลย ด้วยการเรียกชื่อฟังก์ชันนั้นกี่ครั้งต่อกี่ครั้งก็ได้ โดยการเรียกใช้คือ เรียกชื่อฟังก์ชันนั้นตามด้วยวงเล็บ () ซึ่งจะมี Arguments หรือไม่ก็แล้วแต่ฟังก์ชันที่กำหนดไว้

\begin{codelist}{รูปแบบการสร้างฟังก์ชัน}{}
def function_name(args...):
    # statements

def function_name(args...):
    # statements
    return value
\end{codelist}


\begin{codelist}{ตัวอย่างฟังก์ชันที่สร้างขึ้นมาเอง}{}
def happy_birthday_song():
    print('Happy Birthday.')
    print('Happy Birthday to you.')
\end{codelist}

\begin{codelist}{ผลลัพธ์จากตัวอย่างการเรียกใช้ฟังก์ชันที่สร้างขึ้นมาเอง}{}
>>>  happy_birthday_song()
Happy Birthday.
Happy Birthday to you.
\end{codelist}

\begin{codelist}{ตัวอย่างฟังก์ชันที่สร้างขึ้นมาเองโดยมีการใช้ return}{}
def newage(age):
    x = age+50
    return x

age = int(input('Enter your age: '))
print('In 50 years from now, you will be ', newage(age), end='.')
\end{codelist}

\begin{codelist}{ผลลัพธ์จากตัวอย่างฟังก์ชันที่สร้างขึ้นมาเองโดยมีการใช้ return}{}
>>>
Enter your age: 6
In 50 years from now, you will be  56.
>>> 
\end{codelist}


\section{ขอบเขตของตัวแปร}

ในวงเล็บ () เป็นการกำหนด Argument ของฟังก์ชัน ซึ่งจะกลายเป็นพารามิเตอร์ (Parameters) หรือตัวแปรที่ใช้ในฟังก์ชันนั้นๆ เท่านั้นและต้องประมวลผลให้เสร็จสิ้นในฟังก์ชัน หรือเรียกว่า Local Variables ตัวแปร Local นี้ จะไม่สามารถเรียกใช้ที่โปรแกรมหลักได้ และผู้เขียนโปรแกรมก็ไม่สามารถนำตัวแปรนี้ไปใช้ในฟังก์ชันอื่น ๆ ได้ ส่วนตัวแปร Global (Global Variables) จะประกาศไว้ในส่วนหลักของโปรแกรมที่เขียนขึ้น ฟังก์ชันย่อยและตัวโปรแกรมหลักสามารถเรียกใช้ตัวแปร Global ได้

\begin{codelist}{ตัวอย่างการใช้ Local และ Global Variables}{}
def newage(age):
    a = 50 #local variable
    x = age + a #local variable
    return x

age = int(input('Enter your age: ')) #global variable
print('In 50 years from now, you will be ', newage(age), end='.')
\end{codelist}

\begin{codelist}{ผลลัพธ์ตัวอย่างการใช้ Local และ Global Variables}{}
Enter your age: 18
In 50 years from now, you will be  68.
>>> 
\end{codelist}


ตัวแปรใดก็ตามที่จะใช้เป็น Local ให้ใส่คำว่า \texttt{global} ไปด้านหน้าตัวแปรที่ถูกเรียกใช้ในฟังก์ชัน อันที่จริงแล้วจะไม่เขียนคำว่า \texttt{global}  ก็ได้หากชื่อตัวแปรไม่ซ้ำกันเลย

\begin{codelist}{ตัวอย่างการใช้ Global Variables}{}
x = 5
def happy_birthday_song(name):
    global x
    print('Happy Birthday.')
    print('Happy Birthday.')
    print('Happy Birthday.')
    print('Happy Birthday to ' , name)
    print(x)
happy_birthday_song('Mike')
\end{codelist}

\begin{codelist}{ผลลัพธ์ตัวอย่างการใช้ Global Variables}{}
>>>
Happy Birthday.
Happy Birthday.
Happy Birthday.
Happy Birthday to Mike
5
>>>
\end{codelist}

\section{ฟังก์ชัน \texttt{return}}

ฟังก์ชันทุกอันจะต้อง return คือสิ้นสุดการทำงาน แต่ได้เว้นไว้ในฐานที่เข้าใจ เมื่อโปรแกรมไล่การทำงานมาถึงจุด return โปรแกรมจะหยุดทำงานทันทีทั้งๆ ที่ยังมีคำสั่งอื่นตามมาหลังจาก return อีกก็ตาม  ฟังก์ชัน \texttt{return} ซึ่งไม่ได้คืน ค่าอะไรออกมาเรียกว่า Void 

\begin{codelist}{ตัวอย่างการใช้ \texttt{return}}{}
x = 5
def happy_birthday_song(name):
    print('Happy Birthday.')
    return
    print('Happy Birthday.')
    print('Happy Birthday.')
    print('Happy Birthday to ' , name)
    print(x)
happy_birthday_song('Mike')
\end{codelist}

\begin{codelist}{ผลลัพธ์ตัวอย่างการใช้ \texttt{return}}{}
>>>
Happy Birthday.
>>>
\end{codelist}

\section{การคืนค่าจากฟังก์ชัน}

ฟังก์ชันสามารถ return ค่าได้ด้วย ดังเช่นตัวอย่างการหาพื้นที่สี่เหลี่ยม โดยกำหนดฟังก์ชันชื่อ \\  \pyinline{rectangle_area(width, height)} และฟังก์ชันมีพารามิเตอร์สองตัวสำหรับความกว้างและความยาวของสี่เหลี่ยม และฟังก์ชันทำการคืนค่า ผลลัพธ์ที่เป็นพื้นที่กลับไปด้วยคำสั่ง \pyinline{return}


\begin{codelist}{ตัวอย่างการใช้ \texttt{return} ที่มีการส่งค่ากลับ}{}

def happy_birthday_song(name):
    print('Happy Birthday.')
    print('Happy Birthday.')
    print('Happy Birthday.')
    print('Happy Birthday to ' , name)
    return name

def rectangle_area(width, height):
    return width * height
    
x = rectangle_area(4, 3)
print('The area of rectangle is', str(x))
\end{codelist}


\begin{codelist}{ผลลัพธ์ตัวอย่างการใช้ \texttt{return} ที่มีการส่งค่ากลับ}{}
>>>
The area of rectangle is 12
>>>
\end{codelist}


\section{การเขียนโปรแกรมเชิงฟังก์ชัน (Functional Programming)}


การเขียนโปรแกรมเชิงฟังก์ชัน (Functional Programming) เป็นรูปแบบการเขียนโปรแกรมที่เก่าแก่ที่สุดและเป็นรูปแบบที่กลับมาได้รับความนิยมมากในปัจจุบัน คือสร้างฟังก์ชันแล้วให้ฟังก์ชันทำงานร่วมกันโดยไม่มีการใช้ Global Variables เลย ฟังก์ชันหนึ่งทำงานส่งผลลัพธ์แก่อีกฟังก์ชันหนึ่งต่อๆ กันไปเรื่อยๆ ซึ่งภาษา Python สามารถใช้เขียนโปรแกรมแบบนี้ได้ ฟังก์ชันแล้ว return ค่าเป็นผลลัพธ์แก่อีกฟังก์ชันหนึ่งไปเรื่อยๆ  

จากรูป  \pyinline{x=f(g(h(x)))} ทำงานเหมือนกันกับ  \pyinline{x = h(x) }แล้ว  \pyinline{x = g(x)} แล้ว  \pyinline{x = f(x)}

\begin{codelist}{Functional Programming}{}
x = f(g(h(x)))
x = h(x)
x = g(x)
x = f(x)
\end{codelist}


\section{แบบฝึกหัด}

\begin{enumerate} 

\item ตั้งชื่อฟังก์ชัน \pyinline{hello} เพื่อแสดงผลว่า สวัสดีคุณ
\item สร้างฟังก์ชันคำนวณพื้นที่ รับค่า ความกว้าง และ ความยาว
\item สร้างฟังก์ชันชื่อ  \pyinline{maximal_2} รับ arguments 2 ค่า และ return ค่าที่มากที่สุดออกมา
\item สร้างฟังก์ชันชื่อ  \pyinline{maximal_3} รับ arguments 3 ค่า และ return ค่าที่มากที่สุดออกมา
\item สร้างฟังก์ชันรับ arguments 3 ค่า และ return ผลคูณออกมา
\item สร้างฟังก์ชันรับค่าตัวเลขในหน่วยเมตรต่อวินาที แล้ว return ผลในหน่วยกิโลเมตรต่อชั่วโมง
\item สร้างฟังก์ชันหาผลต่างของรายรับกับรายจ่าย และส่งผลกลับมา

\end{enumerate}


