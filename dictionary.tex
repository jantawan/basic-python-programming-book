\chapter{ดิกชันนารี (Dictionary)}
\section{ความหมายของดิกชันนารี}

ดิกชันนารี (Dictionary) คือประเภทข้อมูลที่เก็บข้อมูลในเป็นคู่ๆ ของ Key และ Value โดยที่ Key ใช้สำหรับเป็น Index ในการเข้าถึงข้อมูล Value ของ Key นั้นๆ การสร้าง Dictionary เปล่าจะเขียนอยู่ภายในวงเล็บปีกกา { } หรือ dict() ส่วนการใส่ค่าใน Dictionary จะเขียนอยู่ในวงเล็บปีกกา แต่ละคู่จะขึ้นต้น key ตามด้วย : แล้วตามด้วย value และคั่นคู่ด้วยเครื่องหมาย comma (,) 

\section{การอ่านค่าในดิกชันนารี}

การอ่านค่าใน Dictionary ด้วยคำสั่ง for หรือการใช้ Dictionary เป็น Iterators เมื่อมีการประกาศค่าของ Dictionary แล้ว for จะวนอ่านค่าใน Dictionary ทีละตัว เช่น for key in stocks: print(key, stocks[key])

\section{การหาค่าของคีย์ (Key) ใน Dictionary}
การหาค่าของคีย์ในดิกชันนารีเมื่อรู้ value เช่น สร้างฟังก์ชัน reverse_lookup() มีการส่ง Parameters สองตัวคือ Dictionary กับ Value สำหรับ Key แต่ละตัวใน Dictionary นี้ ถ้า ค่าของ Dictionary เท่ากับ Value ที่ต้องการ ให้ส่งค่า Key กลับมา แต่ถ้าไม่มีก็จะไม่ต้องส่งอะไรออกมา

\section{Dictionary และ List}

keys() method จะแสดง keys ออกมา ส่วน values() method จะแสดง values ออกมา 

ดังนั้น stocks.keys() และ stocks.values() คือ ลิสต์สองตัว ตัวหนึ่งเป็น keys อีกตัวเป็น values ที่ map เข้าหากัน จำไว้ว่า ค่าของลิสต์หรือ dictionary สามารถเป็นได้ทั้งลิสต์และ dictionary ด้วย

\section{ฟังก์ชันที่รับ Parameters ได้ไม่จำกัดจำนวน }

เมื่อไรก็ตามที่ต้องการสร้างฟังก์ชันที่รับ Parameters ได้ไม่จำกัดจำนวนและบอก Keywords ได้ด้วย ให้ใส่เครื่องหมายดอกจัน (*) สองครั้งไว้หน้า Parameters หมายความว่าสิ่งที่ผ่านเข้ามาทาง Parameters จะเป็น Dictionary 

แต่หากต้องการให้ Arguments หรือ Parameters ที่รับเข้ามาเป็น list ให้ใส่เครื่องหมายดอกจัน (*) หนึ่งครั้งไว้หน้า Parameters 

\section{แบบฝึกหัด}

\begin{enumerate} 
\item ให้ dictionary มีค่าคือ {0: 10, 1: 20} เขียนโปรแกรมให้เพิ่มค่าอีกคู่เข้าไป คือ 2:30
\item ทำการรวมค่าทั้งหมดใน dictionary นี้ {'cats':100,'dogs':60,'pigs':300}
\item ทำการเชื่อมต่อ dictionary ทั้ง 3 ชุดนี้ให้เป็นชุดเดียว
	\begin{itemize}
		\item dic1={1:10, 2:20} dic2={3:30, 4:40} dic3={5:50,6:60} 
	\end{itemize}
\item จงดำเนินการต่อไปนี้
	\begin{itemize}
		\item สร้าง dictionary ชื่อ stock มีค่าคือ "banana": 40, "apple": 10, "orange": 15, "pear": 33
		\item สร้าง dictionary ชื่อ prices มีค่าคือ "banana": 4, "apple": 2, "orange": 1.5, "pear": 3
		\item ให้คำนวณว่าถ้าขายผลไม้ได้ทั้งหมดจะได้เงินเท่าไร
	\end{itemize}
\item เขียนฟังก์ชันตรวจสอบว่ามีค่าตัวเลขที่รับมาให้ key ของ dictionary หรือไม่ โดยให้ d คือ dictionary มีค่าคือ {1: 10, 2: 20, 3: 30, 4: 40, 5: 50, 6: 60}

\end{enumerate}
