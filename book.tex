% !TEX program = xelatex
\documentclass[12pt,a4paper]{book}

%% --- Thai --- %
% Set up fonts and encoding
\usepackage{fontspec}
\usepackage{xunicode}
\usepackage{xltxtra}

% Enable line breaks for Thai text
\XeTeXlinebreaklocale "th"
\XeTeXlinebreakskip = 0pt plus 2pt minus 1pt

% Set up Thai fonts
\setmainfont[%
    ItalicFont={TH Sarabun New Italic},%
    BoldFont={TH Sarabun New Bold},%
    BoldItalicFont={TH Sarabun New Bold Italic},%
    Script=Thai,%
    Scale=MatchLowercase,%
    WordSpace=1.25,%
    Mapping=tex-text,%
]{TH Sarabun New}

%% --- %%

\usepackage[a4paper,top=1in,bottom=1in,left=1in,right=1in]{geometry}
\usepackage[hashEnumerators,smartEllipses]{markdown}
\usepackage{indentfirst}

\usepackage{minted}
\setminted{fontsize=\small,baselinestretch=1}
\newminted{py}{linenos=true,frame=lines,framesep=10pt}

\renewcommand{\baselinestretch}{1.5}
\renewcommand{\contentsname}{สารบัญ}
\renewcommand{\chaptername}{บทที่}

\begin{document}
\frontmatter
\title{
เอกสารประกอบการสอนรายวิชา\\
477-201 การเขียนโปรแกรมคอมพิวเตอร์\\
\vspace{1cm}
การเขียนโปรแกรมภาษา Python เบื้องต้น\\
(Basic Python Programming)
}
\author{
ดร. จันทวรรณ ปิยะวัฒน์\\
\vspace{0.25cm}\\
สาขาวิชาระบบสารสนเทศทางธุรกิจ\\
ภาควิชาบริหารธุรกิจ คณะวิทยาการจัดการ\\
มหาวิทยาลัยสงขลานครินทร์\\
ภาคการศึกษาที่ 1/2562
}
\date{\vspace{1ex}}
\maketitle
\chapter{คำนำ}

\begin{markdown}

เอกสารประกอบการสอนเล่มนี้ จัดทำขึ้นสำหรับการสอนรายวิชา 477-201 การเขียนโปรแกรม คอมพิวเตอร์ (Computer Programming) ในภาคการศึกษาที่ 1 ปีการศึกษา 2562 ซึ่งเป็นรายวิชาบังคับของนักศึกษาหลักสูตรระบบสารสนเทศทางธุรกิจ ชั้นปีที่ 2 ภาควิชาบริหารธุรกิจ คณะวิทยาการจัดการ มหาวิทยาลัยสงขลานครินทร์ วิทยาเขตหาดใหญ่ จำนวน 3 หน่วยกิต 3(2-2-5) เป็นการสอนทฤษฎี 2 ชั่วโมงต่อสัปดาห์ ปฏิบัติ 2 ชั่วโมงต่อสัปดาห์ และนักศึกษาควรศึกษาค้นคว้าด้วยตัวเอง 5 ชั่วโมงต่อสัปดาห์

วิชา 477-201 การเขียนโปรแกรมคอมพิวเตอร์ มีจุดมุ่งหมายให้นักศึกษาได้มีพื้นฐาน ความรู้ความเข้าใจในหลักการเขียนโปรแกรมคอมพิวเตอร์เบื้องต้นด้วยภาษา Python ส่วนประกอบ ต่างๆของโปรแกรมคอมพิวเตอร์และภาษา Python สามารถเขียนโปรแกรมคอมพิวเตอร์พื้นฐาน ด้วยภาษา Python ได้ตามการวิเคราะห์และออกแบบขั้นตอนการทำงานของโปรแกรมอย่างมีระบบ สามารถเขียนโปรแกรมแบบมีเงื่อนไขเพื่อการตัดสินใจ เขียนคำสั่งเพื่อให้โปรแกรมทำงานวนซ้ำได้ และเข้าใจการใช้งานโมดูลส่วนเสริมต่างๆ ของโปรแกรมภาษา Python เพื่อนำความรู้เหล่านี้ไปใช้ ในการเขียนโปรแกรมระดับในระดับที่ยากขึ้น ซึ่งได้แก่ การเขียนโปรแกรมแบบฟังก์ชันและ การเขียนโปรแกรมเชิงวัตถุได้

หนังสือเล่มนี้ได้จัดแบ่งเนื้อหาออกเป็น 11 บท ในแต่ละบทจะมีแบบฝึกหัดท้ายบทเพื่อให้ผู้เรียน ได้ลองวิเคราะห์และออกแบบแนวทางแก้ไขปัญหาและพัฒนาออกมาเป็นโปรแกรมด้วยภาษา Python ที่ได้เรียนรู้ไปแล้วได้ ทั้งนี้ผู้จัดทำหวังเป็นอย่างยิ่งว่าเอกสารประกอบการสอนฉบับนี้จะให้ความรู้และเป็นประโยชน์แก่ผู้เรียนและผู้อ่านทุกๆ ท่าน เพื่อสร้างความรู้ความเข้าใจในการฝึกเขียนโปรแกรมคอมพิวเตอร์เบื้องต้นให้ดียิ่งขึ้น หากมีข้อเสนอแนะประการใด ผู้จัดทำขอรับไว้ด้วยความขอบพระคุณยิ่ง

\end{markdown}

\vspace{1cm}
\noindent
\textbf{ดร.จันทวรรณ ปิยะวัฒน์}\\
สาขาวิชาระบบสารสนเทศทางธุรกิจ\\
ภาควิชาบริหารธุรกิจ คณะวิทยาการจัดการ\\
มหาวิทยาลัยสงขลานครินทร์
\tableofcontents

\mainmatter
\chapter{ความรู้เบื้องต้นเกี่ยวกับ Python}

\section{Python คืออะไร}

ในปี ค.ศ. 1980 Mr. Guido van Rossum ได้พัฒนาภาษาโปรแกรมมิ่งขึ้นมาและให้ชื่อว่าภาษา Python และเผยแพร่ให้ใช้งานสู่สาธารณชนในปี ค.ศ. 1991 (Guido, 2019) Python เป็นภาษาโปรแกรมคอมพิวเตอร์ระดับสูงซึ่งไวยากรณ์ของภาษาระดับสูงนี้จะใกล้เคียงคำในภาษาอังกฤษทั่วไป (Downey, 2015) Python ถูกใช้ในการสร้างโมบายแอพพลิเคชั่น เว็บไซต์ เว็บแอพพลิเคชั่น ออนไลน์เซอร์วิส รวมทั้งใช้ใน การวิเคราะห์ข้อมูลและคำนวณ ทางคณิตศาสตร์และวิทยาศาสตร์อย่างแพร่หลาย  ตัวอย่าง ออนไลน์เซอร์วิสที่พัฒนาขึ้นด้วยภาษา Python ได้แก่ Instagram, Uber, Pinterest, Reddit, Spotify และ Dropbox (Shuup, 2019) โดยในระยะหลายปี ที่ผ่านมานี้ Python ได้รับความนิยมสูงขึ้นเรื่อยๆ โดยในเดือนมิถุนายน 2562 ดัชนีความนิยมภาษาโปรแกรมมิ่ง TIOBE ได้แสดงให้เห็นว่า Python เป็นภาษาโปรแกรมมิ่ง ที่ได้รับความนิยมเป็นอันดับที่ 3 เทียบกับภาษาโปรแกรมมิ่งอื่นๆ และมีความนิยมสูงสุดในรอบ 19 ปี (TIOBE, 2019)

หากเปรียบเทียบกับภาษาโปรแกรมมิ่งอื่นๆ แล้ว Python มีไวยากรณ์ภาษา (Syntax) ที่สามารถอ่านง่าย เข้าใจได้ง่าย และเรียนรู้ง่าย Python จึงเป็นภาษาที่เหมาะสมสำหรับการสอนการเขียนโปรแกรมโดยเฉพาะอย่างยิ่งในระดับเบื้องต้น อีกทั้งยังเป็นภาษาที่ยืดหยุ่นสามารถพัฒนาได้บน ระบบปฏิบัติการที่หลากหลาย อาทิ  Windows, Linux, OS/2, MacOS, iOS และ Android นอกจากนี้ โปรแกรมเมอร์ทั่วโลกได้พัฒนาไลบรารี (Libraries) ขึ้นมาจำนวนมากสำหรับต่อยอด การทำงานของภาษา Python พื้นฐาน เช่น Django, Numpy, Pandas, Matplotlib, Flask, Web2py เป็นต้น (Foundation, 2019)

\section{Python ทำงานอย่างไร}

ภาษาโปรแกรมมิ่งระดับสูงจะต้องถูกโปรแกรมแปลภาษา เช่น คอมไพเลอร์ (Compiler) หรือ อินเทอร์พรีเตอร์ (Interpreter) ทำการแปลภาษาระดับสูงให้กลายเป็นภาษาเครื่องที่คอมพิวเตอร์เข้าใจก่อน (Lutz, 2013) ภาษาตระกูลที่ต้องใช้ Compiler เพื่อแปลงเป็นภาษาคอมพิวเตอร์ซึ่งเป็นภาษาที่มนุษย์อ่านไม่ออกแล้วจึงจะทำงานได้ เช่น ภาษา Java ภาษา C หรือภาษา C++ ภาษาพวกนี้จะได้โปรแกรมที่ทำงานรวดเร็วมาก แต่ก็ยากที่จะเรียนรู้ในช่วงการฝึกฝนการเขียน Programming ใหม่ๆ (Barry, 2016)

แต่สำหรับภาษา Python เมื่อได้ Source code ที่เป็นนามสกุลไฟล์ \texttt{.py} แล้ว โปรแกรมจะถูกคอมไพล์โดยคอมไพเลอร์ของ Python เพื่อแปลคำสั่ง Python ให้เป็นคำสั่งแบบ Bytecode และบันทึกไว้ในไฟล์นามสกุล \texttt{.pyc} ต่อมาเมื่อผู้ใช้ต้องการ Run ไฟล์นี้ อินเทอร์พรีเตอร์ (Interpreter) ก็จะแปลง Bytecode เป็นภาษาเครื่องสำหรับการดำเนินการโดยตรงบนฮาร์ดแวร์ (Beazley \& Jones, 2013) อาจเรียกได้ว่า Python เป็นภาษาลูกครึ่งและเรียนรู้ได้ง่าย เหตุผลที่ Python ทำการคอมไพล์เป็น Bytecode เป็นรหัสกลางไว้ก่อนนั้น นั่นก็เพราะ Python ถูกออกแบบมาให้เป็นภาษาการเขียนโปรแกรมที่ไม่ขึ้นกับแพลตฟอร์ม ซึ่งหมายความว่ามีการเขียนโปรแกรมหนึ่งครั้ง แต่สามารถเรียกใช้งานบนอุปกรณ์ใดก็ได้ แต่จะต้องติดตั้ง Python เวอร์ชันที่เหมาะสม 
\begin{markdown}

# ส่วนประกอบต่าง ๆ ของภาษา Python

## ตัวแปร (Variables)

ตัวแปร (Variables) คือชื่อที่กำหนดขึ้นสำหรับใช้เก็บค่าในหน่วยความจำของเครื่องคอมพิวเตอร์

## การตั้งชื่อตัวแปร

การตั้งชื่อตัวแปรมีเงื่อนไขดังนี้

1. ให้ขึ้นต้นด้วยอักษรตัวภาษาอังกฤษตัวใหญ่หรือตัวเล็กตั้งแต่ Aa ถึง Zz เท่านั้น 
1. ประกอบด้วยตัวอักษรหรือตัวเลข 0 ถึงเลข 9 หรือตัวขีดล่าง Underscore (_) แต่ห้ามมีช่องว่าง
1. ตัวเลข 0-9 จะนำหน้าชื่อตัวแปรไม่ได้
1. ตัวพิมพ์เล็กและตัวพิมพ์ใหญ่เป็นตัวแปรคนละตัวกัน (Case-Sensitive) เช่น Name ไม่ใช่ตัวแปรเดียวกันกับ name และจะใช้ใส่เครื่องหมาย = ในการตั้งตัวแปรหรือให้ค่าแก่ตัวแปร นอกจากนี้การตั้งชื่อตัวแปรควรตั้งอย่างสมเหตุสมผล อีกทั้ง ภาษา Python จะมีคำที่ถูกสงวนไว้ในการเขียนโปรแกรม หรือ Keywords ซึ่งห้ามนำมาใช้ในการตั้งชื่อตัวแปร ชื่อฟังก์ชัน หรือ ชื่อคลาส มีดังต่อไปนี้ $\cite{lutz2014}$


\end{markdown}

\cite{lutz2014}

%\begin{minted}{python}
\begin{pycode}
>>> a
1
>>> id(a)
1538021648
\end{pycode}
%\end{minted}
\end{document}