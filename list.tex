\chapter{ลิสต์ (Lists)}
\section{ความหมายของลิสต์}

ลิสต์ (List) เป็นโครงสร้างข้อมูลที่สำคัญมากของภาษา Python คล้ายคลึงกับ Array ในภาษาอื่นๆ คือ ชุดของข้อมูลที่เรียงลำดับต่อๆ กัน จะเป็นค่าของอะไรก็ได้ การสร้าง list โดยกำหนดสัญลักษณ์ [ ] เช่น t = [ ] หรือ t=list() 

ตัวอย่างของการกำหนดค่าใน List เช่น t = [1, 2, 3, 4] เป็น list ของตัวเลข t = [1, "yes", "no", 1.1] เป็นลิสต์ผสมของตัวเลขและข้อความ และ t = [1, 2, 3, ['yes', 'no'], []] เป็นลิสต์ที่มีลิสต์เป็นองค์ประกอบอยู่ด้วย กล่าวได้ว่าเราสามารถเอาค่าของหลายๆ ประเภทมาอยู่รวมกันในลิสต์เดียวกันได้ 

\section{การเข้าถึงค่าในลิสต์}

ค่าในลิสต์จะเรียงตามตัวชี้หรือ Index ที่เริ่มต้นที่ 0 เช่น t=[0] และหากมีลิสต์ซ้อนอยู่ในลิสต์จะสามารถแสดงตัวชี้ได้ด้วยการกำหนดตัวชี้หลักตามด้วยตัวชี้ย่อย เช่น t=[1][0]

\section{การแบ่งข้อมูลในลิสต์ (List Slicing)}

List slicing หรือการแบ่งข้อมูลในลิสต์เป็นชุดข้อมูลย่อยๆ จะเขียนในรูปแบบ [a:b] เมื่อ a เป็น Index เริ่มต้นและ b เป็น Index ก่อนสมาชิกตัวสุดท้ายที่ต้องการตัด

\section{Lists เปลี่ยนแปลงค่าได้}
ลิสต์สามารถเปลี่ยนแปลงค่าได้ 

\section{การใช้ in กับลิสต์}

การดำเนินการด้วย in สามารถใช้กับลิสต์ได้ 

\section{การเดินทางไปในลิสต์ (List Traversal)}
\subsection{การใช้ for loop และตัวดำเนินการ in ในการเดินทางไปในลิสต์ }
\subsection{การใช้ฟังก์ชัน range() กับลิสต์}

สำหรับ range() ของความยาวของตัวแปร จะถูกนำมาใช้ในการเดินทางด้วย index ในลิสต์

\section{ตัวดำเนินการของลิสต์ (List Operators)}

ถ้าต้องการเอาลิสต์มารวมกันให้ใช้เครื่องหมายบวก (+) ถ้าต้องการขยายลิสต์ให้มีค่าชุดเดิมเพิ่มเป็นกี่เท่าตัวให้ใช้เครื่องหมายดอกจัน (*)

\section{เมธอดของลิสต์ (List Methods)}

การขยายลิสต์ด้วยอีกลิสต์หนึ่ง ให้ใช้คำสั่ง list.extend()

การเพิ่มค่าในลิสต์ทำได้ด้วยคำสั่ง list.append() ค่าใหม่ที่ได้จะต่อท้ายตัวสุดท้ายในลิสต์

การเพิ่มค่าในลิสต์โดยกำหนดตำแหน่งของ  index ให้ใช้คำสั่ง list.insert()

การลบค่าในลิสต์ทำได้ด้วยคำสั่ง list.remove()

นอกจากนี้สามารถลบค่าในลิสต์ได้ด้วยคำสั่ง del โดยจะต้องระบุค่า Index ของตัวที่ต้องการลบ เช่น del a[0] เป็นการลบค่าลิสต์ที่มี index 0 หรือถ้าลบเป็นช่วงให้ระบุตำแหน่งเริ่มต้นที่จะลบจนถึงตำแหน่งตัวก่อนสุดท้ายที่จะลบ del a[2:4] จะลบตัวที่ 2 และ 3 ในลิสต์ และถ้าลบค่าในลิสต์ออกทั้งหมดให้ใช้คำสั่ง del a[ : ]

ศึกษาเรื่อง list methods เพิ่มเติมได้ที่ \url{https://docs.python.org/3/tutorial/datastructures.html}

\section{Map, reduce, and filter}

การสร้าง Map ฟังก์ชัน เป็นการทำให้ลิสต์หนึ่งเป็นอีกลิสต์หนึ่ง ให้ฟังก์ชันชื่อ capitalize() รับลิสต์ t มา แล้ว return ลิสต์ r แล้วทำการเดินทางในค่าแต่ละค่า เมื่อเจอค่าก็ให้ทำการประมวลผลเป็นตัวพิมพ์ใหญ่ แล้วเก็บไว้ในลิสต์ r

การสร้าง Reduce ฟังก์ชันเป็นการประมวลผลลิสต์เพื่อการผลสรุป ให้ฟังก์ชันชื่อ sum() รับลิสต์ t มา แล้ว return ค่า sum โดยกำหนดตัวแปรชื่อ sum ให้ค่าเป็น 0 แล้วเดินทางไปในลิสต์ทีละค่า แล้วนำค่าเมื่อบวกกัน เมื่อบวกจนครบทุกตัวแล้วให้ส่งค่ากลับมาเป็น sum

ฟังก์ชันอีกประเภทหนึ่งเรียกว่า filter ฟังก์ชัน คือการ search แบบรับลิสต์มาแล้ว return ค่า คือมีการค้นหาแล้วทำการประมวลผลค่านั้นๆ หรืออาจจะ return มาเป็นลิสต์ก็ได้ แล้วก็ทำการประมวลผลกับข้อมูลที่อยู่ในลิสต์นั้น เช่น ฟังก์ชันรับลิสต์ t เข้าไป แล้ว return ลิสต์ที่เป็น integer เท่านั้น โดยตั้งต้นสร้างลิสต์ r แล้วเดินทางไปในค่าแต่ละค่าของลิสต์ t ถ้าเจอว่าประเภทของค่าเป็น int ให้ทำการแทรกค่าในลิสต์ r เมื่อทำครบแล้วให้ส่งค่าลิสต์ r ออกมา

\section{Lists กับ String}

String เปลี่ยนแปลงค่าไม่ได้ แต่ List เปลี่ยนแปลงค่าได้ ทั้ง String และ List เป็นการเรียงลำดับและเปลี่ยนแปลงค่ากลับกันไปมาได้ด้วยการใช้ฟังก์ชันของลิสต์ เช่น split(), join()

\section{Objects and values}

ภาษา Python เพื่อประหยัดพื้นที่ในหน่วยความจำ สำหรับ String ซึ่งไม่สามารถเปลี่ยนแปลงค่าได้ หรือ Immutable Data Structure ภาษา Python จะชี้ชื่อตัวแปรไปที่ที่เดียวกันสำหรับค่าที่เหมือนกัน 

แต่สำหรับลิสต์ซึ่งเปลี่ยนแปลงค่าได้ หรือ Mutable Data Structure ภาษา Python จะเก็บไว้คนละที่ในหน่วยความจำ แต่สามารถสร้างชื่อตัวแปรต่อๆ กันมาได้ สำหรับตำแหน่งหนึ่งๆ เรียกว่า การทำ Aliasing

Objects อยู่ในหน่วยความจำมี Values แต่ไม่มีชื่อ แต่มี id หรือหมายเลขกำกับ สร้างตัวแปรเพื่อชี้ไปยัง Objects เหล่านั้น ตั้งตัวแปรหลายตัวหรือตัวเดียวก็ได้

\section{แบบฝึกหัด}
\begin{enumerate} 
\item กำหนดคะแนนของนักเรียน 5 คน เก็บไว้ใน list คือ 75 80 68 82 62 ต้องการหาคะแนนรวม คะแนนเฉลี่ย คะแนนมากสุด คะแนนน้อยสุด 
\item ให้ข้อมูลเป็น list มีค่าคือ 3 4 12 31 ให้หาจำนวนสมาชิกใน list  และพิมพ์สมาชิกทุกตัวตัวละบรรทัด
\item ให้ข้อมูลเป็น list มีค่าคือ 25 4 3 15 21 นำมาเรียงจากน้อยไปหามาก พร้อมหาผลคูณของสมาชิกทุกตัว
\item ให้ข้อมูลเป็น list มีค่าคือ 6 9 8 7 10 ให้ลบข้อมูลตัวแรกทิ้งแล้วเพิ่มข้อมูลตัวแรกไปที่ตำแหน่งสุดท้าย
\item จงสร้างข้อมูลเป็น list ชื่อ a มีค่าคือ 1-10 และ list ชื่อ b มีค่าคือ 11-20 โดยใช้ for และใช้ฟังก์ชัน append เพื่อเพิ่มสมาชิกใน list แล้วหาค่า a+b
\end{enumerate}


