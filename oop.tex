\chapter{Object-Oriented Programming (OOP)}
\section{ความหมายของ OOP (Object-Oriented Programming)}

OOP (Object-Oriented Programming) หมายถึงการเขียนโปรแกรมเชิงวัตถุหรือเขียนโปรแกรมแบบออบเจ็กต์ ซึ่งมีวิธีการของการจัดโครงสร้างโปรแกรมเพื่อให้คุณสมบัติและพฤติกรรมหรือการกระทำรวมอยู่ในแต่ละวัตถุ \cite{Mar11} ตัวอย่างเช่น วัตถุอาจเป็นตัวแทนของบุคคลที่มีคุณสมบัติ คือ ชื่อ นามสกุล อายุ ที่อยู่ และมีพฤติกรรม เช่น การเดิน การพูด การหายใจ และการวิ่ง เป็นต้น

\section{คลาส (Classes) และ ออบเจ็กต์ (Objects)}

คลาส (Classes) เปรียบเสมือนแบบพิมพ์เขียวหรือแม่พิมพ์ที่ใช้สร้างออบเจ็กต์ (Objects) ในคลาสจะกำหนดรูปแบบของข้อมูลหรือแอตทริบิวต์หรือตัวแปร (Attributes/Properties/Characteristics) และเมธอด (Methods/Behaviors) ตัวอย่างเช่น คลาสชื่อ \pyinline{pet} มีแอตทริบิวต์คือ \pyinline{legs} ส่วน \pyinline{wolf} คือ ออบเจ็กต์ที่สร้างขึ้น มาจากคลาส \pyinline{pet} ดังตัวอย่างต่อไปนี้

\begin{codelist}{คลาสและออบเจ็กต์}{}
class Pet: # define a class pet
    legs = 0  # a property for the class
    wolf = Pet() # create an object of the class 
    wolf.legs # access to a property of the object
\end{codelist}

\section{การสร้างคลาส}

การสร้างคลาสในภาษา Python ต้องใช้คำสั่ง class ตามด้วยชื่อคลาส ดังตัวอย่างต่อไปนี้

\begin{codelist}{การสร้างคลาส}{}
class Pet: #define a class pet
    legs = 0 #a property for the class
\end{codelist}


\section{การสร้างออบเจ็กต์}

หลังจากที่สร้างคลาสแล้ว ก็จะสามารถสร้างตัวแปรออบเจ็กต์จากคลาสได้ เช่น \pyinline{wolf = Pet()} โดยออบเจ็กต์นี้จะมีแอตทริบิวต์คือ \pyinline{legs} และสามารถเข้าถึงแอตทริบิวต์ของออบเจ็กต์ได้โดยใช้เครื่องหมายจุด (.) ดังตัวอย่างต่อไปนี้

\begin{codelist}{การสร้างออบเจ็กต์}{}
wolf = Pet() # create an object of the class
print(wolf.legs) # access to a property of the object
\end{codelist}


\section{ฟังก์ชัน \texttt{\_\_init\_\_()}}

ฟังก์ชัน \pyinline{__init__()} เรียกว่าเป็นคอนสตรัคเตอร์  (Constructor) ซึ่งถูกเรียกใช้อัตโนมัติทุกครั้ง ในการกำหนดค่าให้แก่แอตทริบิวต์ของออบเจ็กต์และการดำเนินการต่างๆที่จำเป็นเมื่อต้องสร้างออบเจ็กต์ขึ้นมา และสำหรับพารามิเตอร์ self มีไว้เพื่อการเข้าถึงตัวแปรต่างๆ ที่เป็นของคลาส โดยที่ไม่จำเป็นต้องใช้คำว่า self ก็ได้ สามารถตั้งเป็นคำอื่นได้ ดังตัวอย่างต่อไปนี้

\begin{codelist}{The \_\_init\_\_() Function}{}
class Person:
    name = None
    gender = None
    
    #Define the constructor
    def __init__(self, n, g):
        self.name = n
        self.gender = g
#Main code starts here
p1 = Person('Mike', 'male') # create object p1
p2 = Person('Jan', 'female') # create object p2
\end{codelist}


\section{การสร้างเมธอดของออบเจ็กต์}

เมธอดในออบเจ็กต์ก็คือฟังก์ชันที่เป็นของออบเจ็กต์นั้น ในตัวอย่างเป็นการสร้างเมธอด foo เพื่อทำการ print คำว่า Hello แล้วตามด้วยชื่อ เมื่อออบเจ็กต์จะเรียกใช้เมธอดก็ให้เขียนชื่อออบเจ็กต์นั้นตามด้วยจุดแล้ว จึงตามด้วยชื่อเมธอด ดังตัวอย่างต่อไปนี้

\begin{codelist}{Source code ตัวอย่างการสร้างเมธอดของออบเจ็กต์}{}
class Person:
    name = None
    gender = None
    def __init__(self, n, g):
        self.name = n
        self.gender = g
    def foo(self):
        print('Hello' , self.name)

#Main code starts here
p1 = Person('Mike', 'male')
p2 = Person('Jan', 'female')
p1.foo()
p2.foo()
\end{codelist}

ซึ่งได้ผลลัพธ์ดังนี้
\begin{codelist}{Result จากตัวอย่างการสร้างเมธอดของออบเจ็กต์}{}
Hello Mike
Hello Jan
\end{codelist}


\section{การแก้ไขค่าของแอตทริบิวต์ของออบเจ็กต์}

การแก้ไขค่าของแอตทริบิวต์ของออบเจ็กต์สามารถทำได้เหมือนการกำหนดค่าของตัวแปร ดังตัวอย่างต่อไปนี้


\begin{codelist}{ตัวอย่างการแก้ไขค่าของแอตทริบิวต์ของออบเจ็กต์}{}
>>> p1.name = 'Ton-mike'
>>> p1.foo()
Hello Ton-mike
\end{codelist}


\section{การลบแอตทริบิวต์ของออบเจ็กต์}

การลบแอตทริบิวต์ของออบเจ็กต์ให้ใช้คำสั่ง del ตามด้วยแอตทริบิวต์ของออบเจ็กต์นั้น ดังตัวอย่างต่อไปนี้
\begin{codelist}{ตัวอย่างการลบแอตทริบิวต์ของออบเจ็กต์}{}
>>> del p1.name
>>> p1.foo()
Hello None
>>> 
\end{codelist}

\section{การลบออบเจ็กต์}

การลบออบเจ็กต์ทำได้โดยใช้คำสั่ง \pyinline{del} ตามด้วยชื่อของออบเจ็กต์ได้เลย ดังตัวอย่างต่อไปนี้

\begin{codelist}{ตัวอย่างการลบออบเจ็กต}{}
>>> del p1
>>> p1
Traceback (most recent call last):
  File "<pyshell\#29>", line 1, in <module>
    p1
NameError: name 'p1' is not defined
>>> 
\end{codelist}

\section{การสืบทอดคลาส (Class Inheritance)}
การสืบทอดคลาส (Class Inheritance) คือการที่คลาสหนึ่งสามารถสืบทอดเมธอดและแอตทริบิวต์ ของคลาสนั้นไปยังคลาสอื่นได้ เรียกคลาสที่เป็นฐานหรือให้การสืบทอดนั้นว่าคลาสแม่ (Parent class) และ เรียกคลาสที่ได้รับการสืบทอด ว่าคลาสลูก (Child class) ตัวอย่าง คลาสแม่คือ \pyinline{SchoolMember} ได้มีการ สืบทอดเมธอดและ แอตทริบิวต์ ไปยังคลาสลูกคือคลาส \pyinline{Teacher} หลังจากนั้นคลาสลูกก็จะสามารถใช้เมธอด และแอตทริบิวต์ของคลาสแม่ได้ ดังตัวอย่างต่อไปนี้


\begin{codelist}{Source code ตัวอย่างการสืบทอดคลาส}{}
#Define the class Teacher. It inherits from class SchoolMember
class Teacher(SchoolMember):
    def __init__(self, name, age, salary):
        SchoolMember.__init__(self, name, age)
        self.salary = salary

#Main code
teacher1 = Teacher('Mr. Mike Piyawat', 47, 50000)
teacher2 = Teacher('Mrs. Jane Doe', 55, 120000)

print(teacher1.name)
print(teacher1.age)
print(teacher1.salary)
print('\n')
print(teacher2.name)
print(teacher2.age)
print(teacher2.salary)
\end{codelist}

ซึ่งได้ผลลัพธ์ดังนี้
\begin{codelist}{Result จากตัวอย่างการสืบทอดคลาส}{}
>>>
Mr. Mike Piyawat
47
50000


Mrs. Jane Doe
55
120000
>>> 
\end{codelist}


\section{แบบฝึกหัด}
\begin{enumerate} 
\item 	เขียนคลาสชื่อว่า Pet ประกอบด้วย
	\begin{itemize}
	\item 	คอนสตรัคเตอร์ (Constructor)
	\item 	แอตทริบิวต์ genre
	\item 	แอตทริบิวต์ legs
	\item 	เมธอด \pyinline{start_running} แสดงผลทางหน้าจอว่า “Start running”
	\item 	เมธอด \pyinline{stop_running} แสดงผลทางหน้าจอว่า “Stop running”
	\end{itemize}
\item 	 จากข้อ 1 ให้เขียนโปรแกรมเพื่อสร้างออบเจ็กต์จากคลาส Pet สองออบเจ็กต์ แล้วเรียกใช้เมธอดที่มี
\end{enumerate}