\chapter{Object-Oriented Programming (OOP)}
\section{ความหมายของ OOP (Object-Oriented Programming)}

OOP (Object-Oriented Programming) หมายถึงการเขียนโปรแกรมเชิงวัตถุหรือเขียนโปรแกรมแบบออบเจ็กต์ ซึ่งมีวิธีการของการจัดโครงสร้างโปรแกรมเพื่อให้คุณสมบัติและพฤติกรรมหรือการกระทำรวมอยู่ในแต่ละวัตถุ \cite{Mar11} ตัวอย่างเช่น วัตถุอาจเป็นตัวแทนของบุคคลที่มีคุณสมบัติ คือ ชื่อ นามสกุล อายุ ที่อยู่ และมีพฤติกรรม เช่น การเดิน การพูด การหายใจ และการวิ่ง เป็นต้น

\section{คลาส (Classes) และ ออบเจ็กต์ (Objects)}

คลาส (Classes) เปรียบเสมือนแบบพิมพ์เขียวหรือแม่พิมพ์ที่ใช้สร้างออบเจ็กต์ (Objects) ในคลาสจะกำหนดรูปแบบของข้อมูลหรือแอตทริบิวต์หรือตัวแปร (Attributes/Properties/Characteristics) และเมธอด (Methods/Behaviors) ตัวอย่างเช่น คลาสชื่อ pet มีแอตทริบิวต์คือ legs ส่วน wolf คือ ออบเจ็กต์ที่สร้างขึ้น มาจากคลาส pet 

\section{การสร้างคลาส}

การสร้างคลาสในภาษา Python ต้องใช้คำสั่ง class ตามด้วยชื่อคลาส เช่น

\section{การสร้างออบเจ็กต์}

หลังจากที่สร้างคลาสแล้วก็จะสามารถสร้างตัวแปรออบเจ็กต์จากคลาสได้ เช่น \pyinline{wolf = Pet()} โดยออบเจ็กต์นี้จะมีแอตทริบิวต์คือ legs และสามารถเข้าถึงแอตทริบิวต์ของออบเจ็กต์ได้โดยใช้เครื่องหมายจุด (.) 

\section{ฟังก์ชัน \texttt{\_\_init\_\_()}}

ฟังก์ชัน \pyinline{__init__()} เรียกว่าเป็นคอนสตรัคเตอร์  (Constructor) ซึ่งถูกเรียกใช้อัตโนมัติทุกครั้ง ในการกำหนดค่าให้แก่แอตทริบิวต์ของออบเจ็กต์และการดำเนินการต่างๆที่จำเป็นเมื่อต้องสร้างออบเจ็กต์ขึ้นมา และสำหรับพารามิเตอร์ self มีไว้เพื่อการเข้าถึงตัวแปรต่างๆ ที่เป็นของคลาส โดยที่ไม่จำเป็นต้องใช้คำว่า self ก็ได้ สามารถตั้งเป็นคำอื่นได้

\section{การสร้างเมธอดของออบเจ็กต์}

เมธอดในออบเจ็กต์ก็คือฟังก์ชันที่เป็นของออบเจ็กต์นั้น ในตัวอย่างเป็นการสร้างเมธอด foo เพื่อทำการ print คำว่า Hello แล้วตามด้วยชื่อ เมื่อออบเจ็กต์จะเรียกใช้เมธอดก็ให้เขียนชื่อออบเจ็กต์นั้นตามด้วยจุดแล้ว จึงตามด้วยชื่อเมธอด

\section{การแก้ไขค่าของแอตทริบิวต์ของออบเจ็กต์}

การแก้ไขค่าของแอตทริบิวต์ของออบเจ็กต์สามารถทำได้เหมือนการกำหนดค่าของตัวแปร

\section{การลบแอตทริบิวต์ของออบเจ็กต์}

การลบแอตทริบิวต์ของออบเจ็กต์ให้ใช้คำสั่ง del ตามด้วยแอตทริบิวต์ของออบเจ็กต์นั้น

\section{การลบออบเจ็กต์}

การลบออบเจ็กต์ทำได้โดยใช้คำสั่ง del ตามด้วยชื่อของออบเจ็กต์ได้เลย

\section{การสืบทอดคลาส (Class Inheritance)}
การสืบทอดคลาส (Class Inheritance) คือการที่คลาสหนึ่งสามารถสืบทอดเมธอดและแอตทริบิวต์ ของคลาสนั้นไปยังคลาสอื่นได้ เรียกคลาสที่เป็นฐานหรือให้การสืบทอดนั้นว่าคลาสแม่ (Parent class) และ เรียกคลาสที่ได้รับการสืบทอด ว่าคลาสลูก (Child class) ตัวอย่าง คลาสแม่คือ SchoolMember ได้มีการ สืบทอดเมธอดและ แอตทริบิวต์ ไปยังคลาสลูกคือคลาส Teacher หลังจากนั้นคลาสลูกก็จะสามารถใช้เมธอด และแอตทริบิวต์ของคลาสแม่ได้

\section{แบบฝึกหัด}
\begin{enumerate} 
\item 	เขียนคลาสชื่อว่า Pet ประกอบด้วย
	\begin{itemize}
	\item 	คอนสตรัคเตอร์ (Constructor)
	\item 	แอตทริบิวต์ genre
	\item 	แอตทริบิวต์ legs
	\item 	เมธอด \pyinline{start_running} แสดงผลทางหน้าจอว่า “Start running”
	\item 	เมธอด \pyinline{stop_running} แสดงผลทางหน้าจอว่า “Stop running”
	\end{itemize}
\item 	 จากข้อ 1 ให้เขียนโปรแกรมเพื่อสร้างออบเจ็กต์จากคลาส Pet สองออบเจ็กต์ แล้วเรียกใช้เมธอดที่มี
\end{enumerate}