\chapter{ทูเบิล (Tuple)}
\section{ความหมายของ Tuple}

Tuple จะคล้ายกับ List แต่สิ่งที่แตกต่างกันคือ Tuple นั้นเป็นประเภทข้อมูลที่ไม่สามารถเปลี่ยนแปลงได้ เมื่อไม่ต้องการให้ส่วนใดส่วนหนึ่งของโปรแกรมเผลอไปเปลี่ยน Value ก็ควรใช้ Tuple การสร้าง Tuple นั้น จะอยู่ภายในวงเล็บ () และคั่นค่าแต่ละตัวด้วยเครื่องหมายคอมมา (,) ส่วนการเข้าถึงค่าใน Tuple ใช้ index เหมือนกับ list

\begin{codelist}{การสร้าง Tuple}{}
>>> t = 'a', 'b', 'c'
>>> t
('a', 'b', 'c')
>>> t = 'a',
>>> t
('a',)
>>>
\end{codelist}

\begin{codelist}{Tuple ไม่สามารถเปลี่ยนแปลงได้}{}
>>> t = ('a', 'b', 'c')
>>> t[0]
'a'
>>> t[1]
'b'
>>> t[1:]
('b', 'c')
>>> t[0] = 'z'
Traceback (most recent call last):
    File ''<pyshell#16>'', line 1, in <module>
        t[0] = 'z'
TypeError: 'tuple' object does not support item assignment
>>>
\end{codelist}


\section{การสลับค่าของ Tuple}

การสลับค่าของ Tuple สามารถเขียน a, b = b, a และสามารถกำหนดค่าจาก String มาเป็น Tuple ได้ด้วยคำสั่ง split() เช่น username, domain = 'support@classstart.org'.split('@')

\begin{codelist}{Tuple assignment}{}
>>> username, domain = 'support@classstart.org'.split('@')
>>> username
'support'
>>> domain
'classstart.org'
>>>
\end{codelist}


\section{การเก็บค่าการดำเนินการใน Tuple}

เราสามารถใช้ Tuple ในการเก็บค่าที่ได้จากการดำเนินการได้โดยตรง เช่น  floor, remainder = divmod(7, 3) โดยที่ floor คือ ค่าจำนวนเต็มที่ได้จากการหาร ส่วน remainder คือค่าของเศษที่ได้จากการหาร

\begin{codelist}{การให้ค่ากับมาเป็น Tuple}{}
>>> t = divmod(7,3)
>>> t
(2, 1)
>>> floor, remainder = divmod(7,3)
>>> floor
2
>>> remainder
1
\end{codelist}

Source code:
\begin{codelist}{Source code ตัวอย่างการเขียนฟังก์ชันเพื่อให้ค่ากับมาเป็น Tuple}{}
def split_email(email):
    return email.split('@')
\end{codelist}

Result:
\begin{codelist}{Result ตัวอย่างการเขียนฟังก์ชันเพื่อให้ค่ากับมาเป็น Tuple}{}
>>> username, domain = split_email('support@classstart.org')
>>> username
'support'
>>> domain
'classstart.org'
>>>
\end{codelist}

\section{ฟังก์ชัน \pyinline{list()} เปลี่ยน tuple ให้เป็น list}

\begin{codelist}{ฟังก์ชัน list()}{}
>>> t = (1,2,3,4,5)
>>> t
(1,2,3,4,5)
>>> type(t)
|<class \rq{}tuple\rq{}>|
>>> mylist = list(t)
>>> mylist
[1,2,3,4,5]
>>>
\end{codelist}


\section{Dictionary และ Tuple}

เมธอด items() ของ Dictionary จะให้ค่าเป็น List  ของ Tuples โดย Tuples แต่ละตัวคือ Key และ Value 

\begin{codelist}{items() ของ Dictionary จะให้ค่าเป็น List ของ Tuples}{}
>>> d = {'ppt' : 360, 'scb' : 160}
d
{'ppt' : 360, 'scb' : 160}
>>> d.items()
dict_items([('ppt', 360),('scb', 160)])
\end{codelist}


สรุปความแตกต่างในการใช้ Data Types คือถ้าหากต้องการลำดับของอักษร จะใช้ String ถ้าต้องการลำดับของค่าที่เปลี่ยนแปลงได้จะใช้ List ถ้าต้องการลำดับของค่าที่เปลี่ยนแปลงไม่ได้จะใช้ Tuple ถ้าต้องการคู่ลำดับของ Key กับ Value จะใช้ Dictionary

\section{แบบฝึกหัด}
\begin{enumerate} 
\item 	จงเขียนโปรแกรมสร้าง Tuple มีค่าดังต่อไปนี้ ("tuple", False, 3.2, 1) และแสดงค่าออกมา
\item 	จงเขียนโปรแกรมสร้าง Tuple มีค่าดังต่อไปนี้ 4 8 3 แล้วทำการแยกค่าแต่ละค่าให้เป็นตัวแปรแต่ละตัว แล้วให้คำนวณผลรวมของตัวแปรทั้งหมด
\item 	จงเขียนโปรแกรมสร้าง Tuple มีค่าดังต่อไปนี้ 4 6 2 8 3 1 แล้วให้แปลง Tuple เป็น List และการเพิ่มเลข 30 เข้าไป แล้วให้แปลง list กลับมาเป็น Tuple ทำการ Print Tuple นั้น
\item 	จงเขียนโปรแกรมแปลง Tuple ให้เป็น String โดยให้ Tuple มีค่าคือ ('e', 'x', 'e', 'r', 'c', 'i', 's', 'e', 's')
\item 	จงเขียนโปรแกรมตรวจสอบว่ามีข้อมูลอยู่ใน Tuple หรือไม่ โดยให้ Tuple มีค่าคือ ("w", 3, "r", "e", "s", "o", "u", "r", "c", "e")
\end{enumerate}