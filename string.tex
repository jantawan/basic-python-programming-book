\chapter{การใช้งาน String}
\section{ความหมายของ String}

String คือ ข้อความหรือตัวอักษรที่เรียงต่อๆ กันที่อยู่ในเครื่องหมายคำพูดแบบ Double Quotes เช่น “How are you?” หรือ Single Quotes เช่น ‘How are you?’ 

\section{ฟังก์ชัน len()}

มีฟังก์ชันสำหรับ String อยู่หลายฟังก์ชัน เช่น ฟังก์ชัน len() มีวัตถุประสงค์เพื่อหาความยาวของ String นั้น และสำหรับการระบุตำแหน่งของตัวอักษรแต่ละตัวใน String จะใช้สัญลักษณ์ก้ามปู [ ] โดยตัวชี้หรือ Index จะเริ่มต้นจาก 0 เช่น fruit[0] 

\section{การเดินทางตามตัวชี้ของ String}

การเดินทางไปเรื่อยๆ ตามตัวชี้ของ String โดยในตัวอย่างแรกจะเป็นการใช้ while ส่วนในตัวอย่างถัดมาจะใช้ตัวแปร string เป็น Iterator ซึ่งผลลัพธ์จะออกมาเหมือนกัน


\section{การตัดคำใน String}

%การตัดคำใน String ด้วยตัวชี้ (Index) โดยการตัดคำเป็นส่วนย่อยๆ จะมีรูปแบบการเขียนเป็น [start:end] โดยที่ start เป็นตำแหน่งของ Index เริ่มต้นที่ต้องการ และ end นั้นเป็นตำแหน่งก่อนหน้าตำแหน่งสุดท้ายของตัวอักษรที่ต้องการ 

%หากเขียนเป็น [start:] ระบุจุดเริ่มต้นที่ start ผลลัพธ์จะแสดงยาวไปจนถึงจุดสิ้นสุด และหากเขียนเป็น [:end] ผลลัพธ์ที่ได้จะแสดงอักษรตั้งแต่ตัวแรกหรือตัวชี้ที่ศูนย์ไปจนถึงตัวสิ้นสุดที่ระบุไว้

\section{โครงสร้างข้อมูลที่เปลี่ยนแปลงไม่ได้}

String เป็นโครงสร้างข้อมูลที่เปลี่ยนแปลงไม่ได้ ดังนั้นถ้าหากต้องการสร้าง String ใหม่ก็ต้องสร้างเป็น Object ใหม่เท่านั้น

\section{การค้นหาตัวอักษรใน String}

การเขียนโปรแกรมเพื่อค้นหาตัวอักษรใน String แล้วส่งค่าออกมาเป็นค่าตัวชี้ตัวอักษรใน String สามารถเขียนเป็นตัวอย่างฟังก์ชันดังนี้ คือ 

\begin{itemize}
\item ให้ฟังก์ชันชื่อว่า find มีการส่งค่า str เป็น string และค่า char เป็น character 
\item ตั้งตัวนับ i เริ่มต้นที่ 0
\item ขณะที่ i ยังน้อยกว่าจำนวนตัวอักษรใน string ที่ชื่อว่า str ให้ดำเนินการดังนี้คือ
	\begin{itemize}
		\item ตรวจสอบว่า ถ้าตัวชี้ของตัวอักษรเท่ากับตัวอักษรที่ต้องการแล้ว ให้ส่งค่าตัวนับออกมา
	\end{itemize}
\item เมื่อ while เป็นเท็จแล้ว หรือเมื่อค้นจนครบ character ใน string แล้วให้ส่งค่ากลับคือ -1
\end{itemize}

\section{เมธอดของ String (String Methods)}

การดำเนินการกับ String สามารถศึกษาฟังก์ชันได้ที่ \url{https://docs.python.org/2.4/lib/string-methods.html} เช่น str.upper() ใช้ทำงานเพื่อแปลงตัวอักษรภาษาอังกฤษเป็นตัวพิมพ์ใหญ่

\section{in โอเปอร์เรเตอร์}

ใช้ในการพิสูจน์ค่าแบบ Boolean Expression เช่น ‘t’ in s แปลว่า ตัวอักษรตัว t อยู่ใน string ชื่อว่า s หรือไม่ หรือในทางตรงข้ามเพื่อการตรวจสอบว่าไม่มีหรือไม่ให้ใส่ not in 

\section{การเปรียบเทียบ String}

การเปรียบเทียบ String สามารถใช้สัญลักษณ์  ( > , < , <= , <= , == , !=  ) เพื่อเปรียบเทียบค่าของ String สองชุด โดยดูผลลัพธ์ของการเปรียบเทียบค่าของ ASCII value นั้นๆ

\section{การจัดวางรูปแบบของ String (String Formatting)}

การเปลี่ยนการจัดวางรูปแบบของ String มีสองวิธีคือ แบบ Classic ซึ่งทำได้โดยใส่สัญลักษณ์ + แต่สัญลักษณ์นี้จะมีข้อจำกัดคือไม่สามารถแทรกข้อความระหว่างกันได้ แต่หากใช้สัญลักษณ์ \% จะช่วยแก้ไขข้อจำกัดนี้ได้ เช่น \%s สำหรับ string และ \%d สำหรับตัวเลข 

ส่วนการเปลี่ยนการจัดวางของ String แบบ Modern คือ ใช้ .format และระบุ Parameters ได้มากกว่า 1 ตัว อีกทั้งยังสามารถจัดเรียงลำดับ Parameters สลับก่อนหลังได้ตามสะดวก สามารถศึกษาเพิ่มเติมได้ที่ \url{https://docs.python.org/3/library/string.html#format-examples}

\section{แบบฝึกหัด}
\begin{enumerate} 
\item 	เขียนฟังก์ชันต่อไปนี้
\begin{itemize}
\item 	รับค่าข้อความ “James had had had the cat.”
\item 	นับจำนวนคำว่า had
\end{itemize}
\item 	เขียนฟังก์ชันต่อไปนี้
\begin{itemize}
\item 	รับค่าข้อความ “I intend to live forever, or die trying.”
\item 	แทนค่าคำว่า to ด้วย three
\end{itemize}
\item 	เขียนฟังก์ชันต่อไปนี้
\begin{itemize}
\item 	คำนวณความยาวของ string ที่รับมาจากผู้ใช้
\end{itemize}
\item 	เขียนฟังก์ชันต่อไปนี้
\begin{itemize}
\item 	รับ string มาแล้วเปลี่ยนค่ากลับกันระหว่างตัวอักษรตัวแรกกับตัวสุดท้ายของ string นั้น
\end{itemize}
\end{enumerate}


