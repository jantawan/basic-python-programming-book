 \chapter {ประโยคเงื่อนไขในภาษา Python (Conditional Statements)}

\section{การเปรียบเทียบค่า (Boolean Expressions)}

Boolean Expressions คือ การดำเนินการเปรียบเทียบค่าเพื่อให้ได้ผลลัพธ์ออกมาเป็นถูก (True) หากเงื่อนไขเป็นจริง หรือผิด (False) หากเงื่อนไขเป็นเท็จ เช่น ค่าของ x มากกว่าค่าของ y ใช่หรือไม่ ซึ่งผลลัพธ์จะออกมาเป็นถูกหรือผิด


\begin{table}
\caption{สัญลักษณ์การคำนวณทางคณิตศาสตร์แบบย่อ}
\centering
\begin{tabu}{l l}
 \hline
 สัญลักษณ์ & ความหมาย  \\ [0.5ex] 
 \hline
x==y	& x เท่ากับ y \\
x != y	& x ไม่เท่ากับ y \\
x > y	& x มากกว่า y \\
x < y	& x น้อยกว่า y \\
x >= y & x มากกว่าหรือเท่ากับ y \\
x <= y & x น้อยกว่าหรือเท่ากับ y \\
\end{tabu}
\end{table}

\begin{codelist}{การใช้สัญลักษณ์เปรียบเทียบค่า}{}
>>> a = 5
>>> b = 6
>>> a == b
False
>>> 5 > 6
False
>>> 5 < 6
True
>>> 5 >= 6
False
\end{codelist}


\section{ตัวดำเนินการทางตรรกศาสตร์ (Logical Operators)}

การเปรียบเทียบค่ามากกว่าหนึ่งครั้งเชื่อมต่อกันสามารถดำเนินการได้โดยใช้ตัวดำเนินการทาง ตรรกศาสตร์ (Logical Operators) ซึ่งได้แก่ และ (and) หรือ (or) ไม่ (not) เช่น a > 2 or c > b and c >2

ตัวอย่างการใช้ตัวดำเนินการทางตรรกศาสตร์ในภาษา Python

\begin{codelist}{ตัวอย่างการใช้ and or not}{}
>>> a = -4
>>> b = -2
>>> a > 2 or c > b and c> 2
False
\end{codelist}

\begin{table}
\caption{ตารางผลการใช้ and}
\centering
\begin{tabu}{l l l}
 \hline
Boolean Expression 1 (BE1) & Boolean Expression 1 (BE2) & BE1 and BE2  \\ [0.5ex] 
 \hline
False & False & False \\
False & True & False  \\
True & False & False \\
\textbf{True} & \textbf{True} & \textbf{True} \\
\end{tabu}
\end{table}

\begin{table}
\caption{ตารางผลการใช้ or}
\centering
\begin{tabu}{l l l}
 \hline
Boolean Expression 1 (BE1) & Boolean Expression 1 (BE2) & BE1 or BE2  \\ [0.5ex] 
 \hline
\textbf{False} & \textbf{False} & \textbf{False} \\
False & True & True  \\
True & False & True \\
True & True & True \\
\end{tabu}
\end{table}

\begin{table}
\caption{ตารางผลการใช้ not}
\centering
\begin{tabu}{l l}
 \hline
 Boolean Expression & Not BE1  \\ [0.5ex] 
 \hline
False	& True \\
True	& False \\
\end{tabu}
\end{table}

\section{การใช้คำสั่ง if เพื่อเลือกเงื่อนไข}

เงื่อนไขที่ใช้ในภาษา Python คือ if Statement สิ่งที่ตามหลัง if คือ Boolean Expression เรียกว่า Statement ใหญ่ และใน Statement ใหญ่ ก็มี Statement ย่อย การดูว่า Statement ย่อยอยู่ใน if Statement ใดให้ดูที่การย่อหน้าหรือ Indentation ในภาษา Python การย่อหน้าสำคัญมากจะเป็นการบอกว่าอะไรอยู่ภายในอะไร 

รูปแบบของการใช้งานคำสั่ง if ในภาษา Python โดยถ้าหากเงื่อนไขเป็นจริง ตัวโปรแกรมจะประมวลผลในคำสั่ง if หลังเครื่องหมาย : 

\begin{codelist}{รูปแบบของการใช้งานคำสั่ง if}{}
if expresion:
    #statements
\end{codelist}

ตัวอย่างเช่น ให้แสดงข้อความว่า อายุต่ำกว่าเกณฑ์ ถ้าหากค่าอายุที่รับเข้ามาต่ำกว่า 18 ปี

Source code:
\begin{codelist}{Source code จากโจทย์ตัวอย่าง}{}
age = int(input('Enter your age: `))
if age < 18:
   print('You are underage.')
\end{codelist}

Result:
\begin{codelist}{Result จากโจทย์ตัวอย่าง}{}
Enter your age: 15
You are underage.
\end{codelist}


\section{การใช้ if กับ else}

โครงสร้างคำสั่ง if…else จะดำเนินในบล็อคคำสั่ง else ถ้าหากเงื่อนไขในคำสั่ง if นั้นเป็นเท็จ โดยมีรูปแบบการเขียนดังนี้

\begin{codelist}{รูปแบบของการใช้งานคำสั่ง if-else}{}
if expression:
    # statements
else:
    # statements
\end{codelist}

Source code:
\begin{codelist}{Source code ตัวอย่างการใช้ if...else}{}
x = 15
y = 6
if x > y: print('x is greater than y.')
else: print('x is less than or equal to y.')
\end{codelist}

Result:
\begin{codelist}{Result ตัวอย่างการใช้ if...else}{}
x is greater than y.
\end{codelist}


\section{Chained Expressions}

การใช้ Chained Expressions คือ การใส่ elif ไปเรื่อยๆ และเงื่อนไขสุดท้ายจะต้องใช้ else โดยไม่ต้องการระบุ Boolean Expressions ใดๆ อีกแล้วหลังจากที่ใส่ else

Source code:
\begin{codelist}{Source code ตัวอย่างการเขียน Chained Expressions}{}
x = 6
y = 6
if x > y: print('x is greater than y.')
elif x < y: print('x is less than y.')
else: print('x and y are equal.')
\end{codelist}

Result:
\begin{codelist}{Result ตัวอย่างการเขียน Chained Expressions}{}
x and y are equal.
\end{codelist}


\section{Nested Expressions}

if มี Statement อยู่ข้างในได้ และ el se ก็มี Statement อยู่ข้างในได้เช่นกัน เรียกว่า Nested Expressions

Source code:
\begin{codelist}{Source code ตัวอย่างการเขียน Nested Expressions}{}
x = 7
y = 6
if x == y: print('x and y are equal.')
else:
    if x < y: print('x is less than y.')
    else: print('x is greater than y.')
\end{codelist}

Result:
\begin{codelist}{Result  ตัวอย่างการเขียน Nested Expressions}{}
x is greater than y.
\end{codelist}

\section{แบบฝึกหัด}

\begin{enumerate} 
\item  จงเขียน code ต่อไปนี้
\begin{itemize}
\item ให้ถามว่า Are you bored? และให้ตอบว่า y หรือ n
\item ถ้าตอบ y ให้พิมพ์ข้อความว่า Let’s go outside.
\end{itemize}
\item จงเขียน code ต่อไปนี้
\begin{itemize}
\item ตั้งค่าตัวแปร var รับค่าเป็นตัวเลขจำนวนเต็มจากผู้ใช้
\item ถ้า var > 100 ให้แสดงผลว่า “The value is over 100.”
\item ถ้า เป็นกรณีอื่นๆ ให้แสดงผลว่า “The value is less than or equal 100.”
\end{itemize}
\item จงเขียน code ต่อไปนี้
\begin{itemize}
\item รับค่า a, b เป็นจำนวนเต็ม
\item แสดงผลว่า a > b หรือ a < b หรือ a = b
\end{itemize}
\item จงเขียน code ต่อไปนี้
\begin{itemize}
\item รับค่า score เป็นจุดทศนิยม
\item ถ้าคะแนน 81-100 แสดงผลว่า เกรด A
\item ถ้าคะแนน 61-80  แสดงผลว่า เกรด B
\item ถ้าคะแนน 41-60 แสดงผลว่า เกรด C
\item ถ้าคะแนน 0-40 แสดงผลว่า เกรด F
\item เมื่อแสดงผลดังกล่าวแล้ว ให้แจ้งด้วยว่า “ตัดเกรดแล้ว”
\end{itemize}
\end{enumerate}
